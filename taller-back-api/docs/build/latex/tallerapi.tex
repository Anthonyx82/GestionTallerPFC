%% Generated by Sphinx.
\def\sphinxdocclass{report}
\documentclass[letterpaper,10pt,spanish]{sphinxmanual}
\ifdefined\pdfpxdimen
   \let\sphinxpxdimen\pdfpxdimen\else\newdimen\sphinxpxdimen
\fi \sphinxpxdimen=.75bp\relax
\ifdefined\pdfimageresolution
    \pdfimageresolution= \numexpr \dimexpr1in\relax/\sphinxpxdimen\relax
\fi
%% let collapsible pdf bookmarks panel have high depth per default
\PassOptionsToPackage{bookmarksdepth=5}{hyperref}

\PassOptionsToPackage{booktabs}{sphinx}
\PassOptionsToPackage{colorrows}{sphinx}

\PassOptionsToPackage{warn}{textcomp}
\usepackage[utf8]{inputenc}
\ifdefined\DeclareUnicodeCharacter
% support both utf8 and utf8x syntaxes
  \ifdefined\DeclareUnicodeCharacterAsOptional
    \def\sphinxDUC#1{\DeclareUnicodeCharacter{"#1}}
  \else
    \let\sphinxDUC\DeclareUnicodeCharacter
  \fi
  \sphinxDUC{00A0}{\nobreakspace}
  \sphinxDUC{2500}{\sphinxunichar{2500}}
  \sphinxDUC{2502}{\sphinxunichar{2502}}
  \sphinxDUC{2514}{\sphinxunichar{2514}}
  \sphinxDUC{251C}{\sphinxunichar{251C}}
  \sphinxDUC{2572}{\textbackslash}
\fi
\usepackage{cmap}
\usepackage[T1]{fontenc}
\usepackage{amsmath,amssymb,amstext}
\usepackage{babel}



\usepackage{tgtermes}
\usepackage{tgheros}
\renewcommand{\ttdefault}{txtt}



\usepackage[Sonny]{fncychap}
\ChNameVar{\Large\normalfont\sffamily}
\ChTitleVar{\Large\normalfont\sffamily}
\usepackage{sphinx}

\fvset{fontsize=auto}
\usepackage{geometry}


% Include hyperref last.
\usepackage{hyperref}
% Fix anchor placement for figures with captions.
\usepackage{hypcap}% it must be loaded after hyperref.
% Set up styles of URL: it should be placed after hyperref.
\urlstyle{same}

\addto\captionsspanish{\renewcommand{\contentsname}{Contenido:}}

\usepackage{sphinxmessages}
\setcounter{tocdepth}{1}



\title{Taller API}
\date{05 de junio de 2025}
\release{2.0.0}
\author{Antonio Martin Sosa}
\newcommand{\sphinxlogo}{\vbox{}}
\renewcommand{\releasename}{Versión}
\makeindex
\begin{document}

\ifdefined\shorthandoff
  \ifnum\catcode`\=\string=\active\shorthandoff{=}\fi
  \ifnum\catcode`\"=\active\shorthandoff{"}\fi
\fi

\pagestyle{empty}
\sphinxmaketitle
\pagestyle{plain}
\sphinxtableofcontents
\pagestyle{normal}
\phantomsection\label{\detokenize{index::doc}}


\sphinxAtStartPar
Add your content using \sphinxcode{\sphinxupquote{reStructuredText}} syntax. See the
\sphinxhref{https://www.sphinx-doc.org/en/master/usage/restructuredtext/index.html}{reStructuredText}
documentation for details.

\sphinxstepscope


\chapter{Introducción}
\label{\detokenize{introduccion:introduccion}}\label{\detokenize{introduccion::doc}}
\sphinxAtStartPar
Bienvenido a la documentación técnica de la API de gestión de vehículos e informes de diagnóstico.

\sphinxAtStartPar
Este sistema permite registrar usuarios, añadir vehículos, capturar errores OBD\sphinxhyphen{}II (DTC), generar informes personalizados y compartirlos mediante enlaces públicos enviados por correo electrónico.

\sphinxAtStartPar
Tecnologías utilizadas:
\begin{itemize}
\item {} 
\sphinxAtStartPar
\sphinxstylestrong{FastAPI}: para la creación de endpoints RESTful modernos.

\item {} 
\sphinxAtStartPar
\sphinxstylestrong{SQLAlchemy}: ORM para gestión de base de datos relacional.

\item {} 
\sphinxAtStartPar
\sphinxstylestrong{Pydantic}: validación de datos.

\item {} 
\sphinxAtStartPar
\sphinxstylestrong{JWT (JSON Web Tokens)}: autenticación y autorización.

\item {} 
\sphinxAtStartPar
\sphinxstylestrong{FastAPI\sphinxhyphen{}Mail}: envío de correos con informes.

\item {} 
\sphinxAtStartPar
\sphinxstylestrong{MySQL}: base de datos principal.

\item {} 
\sphinxAtStartPar
\sphinxstylestrong{Sphinx}: generación de esta documentación.

\end{itemize}

\sphinxAtStartPar
Esta documentación se divide en las siguientes secciones:
\begin{itemize}
\item {} 
\sphinxAtStartPar
Descripción de los endpoints disponibles.

\item {} 
\sphinxAtStartPar
Modelos de datos (ORM y Pydantic).

\item {} 
\sphinxAtStartPar
Configuración del sistema (DB, correo, JWT, etc.).

\end{itemize}

\sphinxstepscope


\chapter{Endpoints de la API}
\label{\detokenize{endpoints:endpoints-de-la-api}}\label{\detokenize{endpoints::doc}}
\sphinxAtStartPar
A continuación se detallan todos los endpoints disponibles en la API, clasificados por funcionalidad.
\index{module@\spxentry{module}!main@\spxentry{main}}\index{main@\spxentry{main}!module@\spxentry{module}}\index{Base (clase en main)@\spxentry{Base}\spxextra{clase en main}}\phantomsection\label{\detokenize{endpoints:module-main}}

\begin{fulllineitems}
\phantomsection\label{\detokenize{endpoints:main.Base}}
\pysigstartsignatures
\pysiglinewithargsret
{\sphinxbfcode{\sphinxupquote{\DUrole{k}{class}\DUrole{w}{ }}}\sphinxcode{\sphinxupquote{main.}}\sphinxbfcode{\sphinxupquote{Base}}}
{\sphinxparam{\DUrole{o}{**}\DUrole{n}{kwargs}\DUrole{p}{:}\DUrole{w}{ }\DUrole{n}{Any}}}
{}
\pysigstopsignatures
\sphinxAtStartPar
Bases: \sphinxcode{\sphinxupquote{object}}

\sphinxAtStartPar
Configuración del sistema de envío de correos (FastAPI Mail):
\begin{itemize}
\item {} 
\sphinxAtStartPar
Las credenciales y parámetros se cargan desde variables de entorno.

\item {} 
\sphinxAtStartPar
\sphinxtitleref{FastMail} se instancia con esta configuración para ser usado en envíos.

\end{itemize}
\index{metadata (atributo de main.Base)@\spxentry{metadata}\spxextra{atributo de main.Base}}

\begin{fulllineitems}
\phantomsection\label{\detokenize{endpoints:main.Base.metadata}}
\pysigstartsignatures
\pysigline
{\sphinxbfcode{\sphinxupquote{metadata}}\sphinxbfcode{\sphinxupquote{\DUrole{p}{:}\DUrole{w}{ }MetaData}}\sphinxbfcode{\sphinxupquote{\DUrole{w}{ }\DUrole{p}{=}\DUrole{w}{ }MetaData()}}}
\pysigstopsignatures
\end{fulllineitems}

\index{registry (atributo de main.Base)@\spxentry{registry}\spxextra{atributo de main.Base}}

\begin{fulllineitems}
\phantomsection\label{\detokenize{endpoints:main.Base.registry}}
\pysigstartsignatures
\pysigline
{\sphinxbfcode{\sphinxupquote{registry}}\sphinxbfcode{\sphinxupquote{\DUrole{p}{:}\DUrole{w}{ }registry}}\sphinxbfcode{\sphinxupquote{\DUrole{w}{ }\DUrole{p}{=}\DUrole{w}{ }\textless{}sqlalchemy.orm.decl\_api.registry object\textgreater{}}}}
\pysigstopsignatures
\end{fulllineitems}


\end{fulllineitems}

\index{ErrorVehiculo (clase en main)@\spxentry{ErrorVehiculo}\spxextra{clase en main}}

\begin{fulllineitems}
\phantomsection\label{\detokenize{endpoints:main.ErrorVehiculo}}
\pysigstartsignatures
\pysiglinewithargsret
{\sphinxbfcode{\sphinxupquote{\DUrole{k}{class}\DUrole{w}{ }}}\sphinxcode{\sphinxupquote{main.}}\sphinxbfcode{\sphinxupquote{ErrorVehiculo}}}
{\sphinxparam{\DUrole{o}{**}\DUrole{n}{kwargs}}}
{}
\pysigstopsignatures
\sphinxAtStartPar
Bases: {\hyperref[\detokenize{modelos:main.Base}]{\sphinxcrossref{\sphinxcode{\sphinxupquote{Base}}}}}

\sphinxAtStartPar
Modelo ORM que almacena los errores OBD\sphinxhyphen{}II (códigos DTC) de un vehículo.
\begin{description}
\sphinxlineitem{Atributos:}
\sphinxAtStartPar
id (int): ID del error.
vehiculo\_id (int): ID del vehículo asociado.
codigo\_dtc (str): Código de diagnóstico (ej. P0301).

\sphinxlineitem{Relaciones:}
\sphinxAtStartPar
vehiculo (Vehiculo): Vehículo asociado.

\end{description}
\index{codigo\_dtc (atributo de main.ErrorVehiculo)@\spxentry{codigo\_dtc}\spxextra{atributo de main.ErrorVehiculo}}

\begin{fulllineitems}
\phantomsection\label{\detokenize{endpoints:main.ErrorVehiculo.codigo_dtc}}
\pysigstartsignatures
\pysigline
{\sphinxbfcode{\sphinxupquote{codigo\_dtc}}}
\pysigstopsignatures
\end{fulllineitems}

\index{id (atributo de main.ErrorVehiculo)@\spxentry{id}\spxextra{atributo de main.ErrorVehiculo}}

\begin{fulllineitems}
\phantomsection\label{\detokenize{endpoints:main.ErrorVehiculo.id}}
\pysigstartsignatures
\pysigline
{\sphinxbfcode{\sphinxupquote{id}}}
\pysigstopsignatures
\end{fulllineitems}

\index{vehiculo (atributo de main.ErrorVehiculo)@\spxentry{vehiculo}\spxextra{atributo de main.ErrorVehiculo}}

\begin{fulllineitems}
\phantomsection\label{\detokenize{endpoints:main.ErrorVehiculo.vehiculo}}
\pysigstartsignatures
\pysigline
{\sphinxbfcode{\sphinxupquote{vehiculo}}}
\pysigstopsignatures
\end{fulllineitems}

\index{vehiculo\_id (atributo de main.ErrorVehiculo)@\spxentry{vehiculo\_id}\spxextra{atributo de main.ErrorVehiculo}}

\begin{fulllineitems}
\phantomsection\label{\detokenize{endpoints:main.ErrorVehiculo.vehiculo_id}}
\pysigstartsignatures
\pysigline
{\sphinxbfcode{\sphinxupquote{vehiculo\_id}}}
\pysigstopsignatures
\end{fulllineitems}


\end{fulllineitems}

\index{ErrorVehiculoRegistro (clase en main)@\spxentry{ErrorVehiculoRegistro}\spxextra{clase en main}}

\begin{fulllineitems}
\phantomsection\label{\detokenize{endpoints:main.ErrorVehiculoRegistro}}
\pysigstartsignatures
\pysiglinewithargsret
{\sphinxbfcode{\sphinxupquote{\DUrole{k}{class}\DUrole{w}{ }}}\sphinxcode{\sphinxupquote{main.}}\sphinxbfcode{\sphinxupquote{ErrorVehiculoRegistro}}}
{\sphinxparam{\DUrole{keyword-only-separator}{\DUrole{o}{\sphinxstyleabbreviation{*} (Keyword\sphinxhyphen{}only parameters separator (PEP 3102))}}}\sphinxparamcomma \sphinxparam{\DUrole{n}{codigo\_dtc}\DUrole{p}{:}\DUrole{w}{ }\DUrole{n}{list\DUrole{p}{{[}}str\DUrole{p}{{]}}}}\sphinxparamcomma \sphinxparam{\DUrole{n}{vehiculo\_id}\DUrole{p}{:}\DUrole{w}{ }\DUrole{n}{int}}}
{}
\pysigstopsignatures
\sphinxAtStartPar
Bases: \sphinxcode{\sphinxupquote{BaseModel}}

\sphinxAtStartPar
Modelo de solicitud para registrar errores OBD\sphinxhyphen{}II de un vehículo.
\begin{description}
\sphinxlineitem{Atributos:}
\sphinxAtStartPar
codigo\_dtc (list{[}str{]}): Lista de códigos DTC (códigos de diagnóstico).
vehiculo\_id (int): ID del vehículo al que se le asocian los errores.

\end{description}
\index{codigo\_dtc (atributo de main.ErrorVehiculoRegistro)@\spxentry{codigo\_dtc}\spxextra{atributo de main.ErrorVehiculoRegistro}}

\begin{fulllineitems}
\phantomsection\label{\detokenize{endpoints:main.ErrorVehiculoRegistro.codigo_dtc}}
\pysigstartsignatures
\pysigline
{\sphinxbfcode{\sphinxupquote{codigo\_dtc}}\sphinxbfcode{\sphinxupquote{\DUrole{p}{:}\DUrole{w}{ }list\DUrole{p}{{[}}str\DUrole{p}{{]}}}}}
\pysigstopsignatures
\end{fulllineitems}

\index{model\_config (atributo de main.ErrorVehiculoRegistro)@\spxentry{model\_config}\spxextra{atributo de main.ErrorVehiculoRegistro}}

\begin{fulllineitems}
\phantomsection\label{\detokenize{endpoints:main.ErrorVehiculoRegistro.model_config}}
\pysigstartsignatures
\pysigline
{\sphinxbfcode{\sphinxupquote{model\_config}}\sphinxbfcode{\sphinxupquote{\DUrole{p}{:}\DUrole{w}{ }ClassVar\DUrole{p}{{[}}ConfigDict\DUrole{p}{{]}}}}\sphinxbfcode{\sphinxupquote{\DUrole{w}{ }\DUrole{p}{=}\DUrole{w}{ }\{\}}}}
\pysigstopsignatures
\sphinxAtStartPar
Configuration for the model, should be a dictionary conforming to {[}\sphinxtitleref{ConfigDict}{]}{[}pydantic.config.ConfigDict{]}.

\end{fulllineitems}

\index{vehiculo\_id (atributo de main.ErrorVehiculoRegistro)@\spxentry{vehiculo\_id}\spxextra{atributo de main.ErrorVehiculoRegistro}}

\begin{fulllineitems}
\phantomsection\label{\detokenize{endpoints:main.ErrorVehiculoRegistro.vehiculo_id}}
\pysigstartsignatures
\pysigline
{\sphinxbfcode{\sphinxupquote{vehiculo\_id}}\sphinxbfcode{\sphinxupquote{\DUrole{p}{:}\DUrole{w}{ }int}}}
\pysigstopsignatures
\end{fulllineitems}


\end{fulllineitems}

\index{InformeCompartido (clase en main)@\spxentry{InformeCompartido}\spxextra{clase en main}}

\begin{fulllineitems}
\phantomsection\label{\detokenize{endpoints:main.InformeCompartido}}
\pysigstartsignatures
\pysiglinewithargsret
{\sphinxbfcode{\sphinxupquote{\DUrole{k}{class}\DUrole{w}{ }}}\sphinxcode{\sphinxupquote{main.}}\sphinxbfcode{\sphinxupquote{InformeCompartido}}}
{\sphinxparam{\DUrole{o}{**}\DUrole{n}{kwargs}}}
{}
\pysigstopsignatures
\sphinxAtStartPar
Bases: {\hyperref[\detokenize{modelos:main.Base}]{\sphinxcrossref{\sphinxcode{\sphinxupquote{Base}}}}}

\sphinxAtStartPar
Modelo ORM que representa un informe compartido con un cliente por email.
\begin{description}
\sphinxlineitem{Atributos:}
\sphinxAtStartPar
id (int): ID del informe.
token (str): Token único para acceder al informe.
vehiculo\_id (int): ID del vehículo relacionado.
email\_cliente (str): Email al que se envía el informe.
creado\_en (str): Fecha y hora de creación del informe (ISO format).

\sphinxlineitem{Relaciones:}
\sphinxAtStartPar
vehiculo (Vehiculo): Vehículo asociado.

\end{description}
\index{creado\_en (atributo de main.InformeCompartido)@\spxentry{creado\_en}\spxextra{atributo de main.InformeCompartido}}

\begin{fulllineitems}
\phantomsection\label{\detokenize{endpoints:main.InformeCompartido.creado_en}}
\pysigstartsignatures
\pysigline
{\sphinxbfcode{\sphinxupquote{creado\_en}}}
\pysigstopsignatures
\end{fulllineitems}

\index{email\_cliente (atributo de main.InformeCompartido)@\spxentry{email\_cliente}\spxextra{atributo de main.InformeCompartido}}

\begin{fulllineitems}
\phantomsection\label{\detokenize{endpoints:main.InformeCompartido.email_cliente}}
\pysigstartsignatures
\pysigline
{\sphinxbfcode{\sphinxupquote{email\_cliente}}}
\pysigstopsignatures
\end{fulllineitems}

\index{id (atributo de main.InformeCompartido)@\spxentry{id}\spxextra{atributo de main.InformeCompartido}}

\begin{fulllineitems}
\phantomsection\label{\detokenize{endpoints:main.InformeCompartido.id}}
\pysigstartsignatures
\pysigline
{\sphinxbfcode{\sphinxupquote{id}}}
\pysigstopsignatures
\end{fulllineitems}

\index{token (atributo de main.InformeCompartido)@\spxentry{token}\spxextra{atributo de main.InformeCompartido}}

\begin{fulllineitems}
\phantomsection\label{\detokenize{endpoints:main.InformeCompartido.token}}
\pysigstartsignatures
\pysigline
{\sphinxbfcode{\sphinxupquote{token}}}
\pysigstopsignatures
\end{fulllineitems}

\index{vehiculo (atributo de main.InformeCompartido)@\spxentry{vehiculo}\spxextra{atributo de main.InformeCompartido}}

\begin{fulllineitems}
\phantomsection\label{\detokenize{endpoints:main.InformeCompartido.vehiculo}}
\pysigstartsignatures
\pysigline
{\sphinxbfcode{\sphinxupquote{vehiculo}}}
\pysigstopsignatures
\end{fulllineitems}

\index{vehiculo\_id (atributo de main.InformeCompartido)@\spxentry{vehiculo\_id}\spxextra{atributo de main.InformeCompartido}}

\begin{fulllineitems}
\phantomsection\label{\detokenize{endpoints:main.InformeCompartido.vehiculo_id}}
\pysigstartsignatures
\pysigline
{\sphinxbfcode{\sphinxupquote{vehiculo\_id}}}
\pysigstopsignatures
\end{fulllineitems}


\end{fulllineitems}

\index{InformeRequest (clase en main)@\spxentry{InformeRequest}\spxextra{clase en main}}

\begin{fulllineitems}
\phantomsection\label{\detokenize{endpoints:main.InformeRequest}}
\pysigstartsignatures
\pysiglinewithargsret
{\sphinxbfcode{\sphinxupquote{\DUrole{k}{class}\DUrole{w}{ }}}\sphinxcode{\sphinxupquote{main.}}\sphinxbfcode{\sphinxupquote{InformeRequest}}}
{\sphinxparam{\DUrole{keyword-only-separator}{\DUrole{o}{\sphinxstyleabbreviation{*}}}}\sphinxparamcomma \sphinxparam{\DUrole{n}{email}\DUrole{p}{:}\DUrole{w}{ }\DUrole{n}{str}}}
{}
\pysigstopsignatures
\sphinxAtStartPar
Bases: \sphinxcode{\sphinxupquote{BaseModel}}

\sphinxAtStartPar
Modelo de solicitud para generar y enviar un informe por correo.
\begin{description}
\sphinxlineitem{Atributos:}
\sphinxAtStartPar
email (str): Dirección de email del cliente destinatario.

\end{description}
\index{email (atributo de main.InformeRequest)@\spxentry{email}\spxextra{atributo de main.InformeRequest}}

\begin{fulllineitems}
\phantomsection\label{\detokenize{endpoints:main.InformeRequest.email}}
\pysigstartsignatures
\pysigline
{\sphinxbfcode{\sphinxupquote{email}}\sphinxbfcode{\sphinxupquote{\DUrole{p}{:}\DUrole{w}{ }str}}}
\pysigstopsignatures
\end{fulllineitems}

\index{model\_config (atributo de main.InformeRequest)@\spxentry{model\_config}\spxextra{atributo de main.InformeRequest}}

\begin{fulllineitems}
\phantomsection\label{\detokenize{endpoints:main.InformeRequest.model_config}}
\pysigstartsignatures
\pysigline
{\sphinxbfcode{\sphinxupquote{model\_config}}\sphinxbfcode{\sphinxupquote{\DUrole{p}{:}\DUrole{w}{ }ClassVar\DUrole{p}{{[}}ConfigDict\DUrole{p}{{]}}}}\sphinxbfcode{\sphinxupquote{\DUrole{w}{ }\DUrole{p}{=}\DUrole{w}{ }\{\}}}}
\pysigstopsignatures
\sphinxAtStartPar
Configuration for the model, should be a dictionary conforming to {[}\sphinxtitleref{ConfigDict}{]}{[}pydantic.config.ConfigDict{]}.

\end{fulllineitems}


\end{fulllineitems}

\index{Usuario (clase en main)@\spxentry{Usuario}\spxextra{clase en main}}

\begin{fulllineitems}
\phantomsection\label{\detokenize{endpoints:main.Usuario}}
\pysigstartsignatures
\pysiglinewithargsret
{\sphinxbfcode{\sphinxupquote{\DUrole{k}{class}\DUrole{w}{ }}}\sphinxcode{\sphinxupquote{main.}}\sphinxbfcode{\sphinxupquote{Usuario}}}
{\sphinxparam{\DUrole{o}{**}\DUrole{n}{kwargs}}}
{}
\pysigstopsignatures
\sphinxAtStartPar
Bases: {\hyperref[\detokenize{modelos:main.Base}]{\sphinxcrossref{\sphinxcode{\sphinxupquote{Base}}}}}

\sphinxAtStartPar
Modelo ORM que representa a los usuarios del sistema.
\begin{description}
\sphinxlineitem{Atributos:}
\sphinxAtStartPar
id (int): ID autoincremental (clave primaria).
username (str): Nombre de usuario, único.
password\_hash (str): Contraseña hasheada con bcrypt.

\sphinxlineitem{Relaciones:}
\sphinxAtStartPar
vehiculos (List{[}Vehiculo{]}): Lista de vehículos registrados por el usuario.

\end{description}
\index{id (atributo de main.Usuario)@\spxentry{id}\spxextra{atributo de main.Usuario}}

\begin{fulllineitems}
\phantomsection\label{\detokenize{endpoints:main.Usuario.id}}
\pysigstartsignatures
\pysigline
{\sphinxbfcode{\sphinxupquote{id}}}
\pysigstopsignatures
\end{fulllineitems}

\index{password\_hash (atributo de main.Usuario)@\spxentry{password\_hash}\spxextra{atributo de main.Usuario}}

\begin{fulllineitems}
\phantomsection\label{\detokenize{endpoints:main.Usuario.password_hash}}
\pysigstartsignatures
\pysigline
{\sphinxbfcode{\sphinxupquote{password\_hash}}}
\pysigstopsignatures
\end{fulllineitems}

\index{username (atributo de main.Usuario)@\spxentry{username}\spxextra{atributo de main.Usuario}}

\begin{fulllineitems}
\phantomsection\label{\detokenize{endpoints:main.Usuario.username}}
\pysigstartsignatures
\pysigline
{\sphinxbfcode{\sphinxupquote{username}}}
\pysigstopsignatures
\end{fulllineitems}

\index{vehiculos (atributo de main.Usuario)@\spxentry{vehiculos}\spxextra{atributo de main.Usuario}}

\begin{fulllineitems}
\phantomsection\label{\detokenize{endpoints:main.Usuario.vehiculos}}
\pysigstartsignatures
\pysigline
{\sphinxbfcode{\sphinxupquote{vehiculos}}}
\pysigstopsignatures
\end{fulllineitems}


\end{fulllineitems}

\index{UsuarioLogin (clase en main)@\spxentry{UsuarioLogin}\spxextra{clase en main}}

\begin{fulllineitems}
\phantomsection\label{\detokenize{endpoints:main.UsuarioLogin}}
\pysigstartsignatures
\pysiglinewithargsret
{\sphinxbfcode{\sphinxupquote{\DUrole{k}{class}\DUrole{w}{ }}}\sphinxcode{\sphinxupquote{main.}}\sphinxbfcode{\sphinxupquote{UsuarioLogin}}}
{\sphinxparam{\DUrole{keyword-only-separator}{\DUrole{o}{\sphinxstyleabbreviation{*}}}}\sphinxparamcomma \sphinxparam{\DUrole{n}{username}\DUrole{p}{:}\DUrole{w}{ }\DUrole{n}{str}}\sphinxparamcomma \sphinxparam{\DUrole{n}{password}\DUrole{p}{:}\DUrole{w}{ }\DUrole{n}{str}}}
{}
\pysigstopsignatures
\sphinxAtStartPar
Bases: \sphinxcode{\sphinxupquote{BaseModel}}

\sphinxAtStartPar
Modelo de solicitud para iniciar sesión de usuario.
\begin{description}
\sphinxlineitem{Atributos:}
\sphinxAtStartPar
username (str): Nombre de usuario.
password (str): Contraseña en texto plano.

\end{description}
\index{model\_config (atributo de main.UsuarioLogin)@\spxentry{model\_config}\spxextra{atributo de main.UsuarioLogin}}

\begin{fulllineitems}
\phantomsection\label{\detokenize{endpoints:main.UsuarioLogin.model_config}}
\pysigstartsignatures
\pysigline
{\sphinxbfcode{\sphinxupquote{model\_config}}\sphinxbfcode{\sphinxupquote{\DUrole{p}{:}\DUrole{w}{ }ClassVar\DUrole{p}{{[}}ConfigDict\DUrole{p}{{]}}}}\sphinxbfcode{\sphinxupquote{\DUrole{w}{ }\DUrole{p}{=}\DUrole{w}{ }\{\}}}}
\pysigstopsignatures
\sphinxAtStartPar
Configuration for the model, should be a dictionary conforming to {[}\sphinxtitleref{ConfigDict}{]}{[}pydantic.config.ConfigDict{]}.

\end{fulllineitems}

\index{password (atributo de main.UsuarioLogin)@\spxentry{password}\spxextra{atributo de main.UsuarioLogin}}

\begin{fulllineitems}
\phantomsection\label{\detokenize{endpoints:main.UsuarioLogin.password}}
\pysigstartsignatures
\pysigline
{\sphinxbfcode{\sphinxupquote{password}}\sphinxbfcode{\sphinxupquote{\DUrole{p}{:}\DUrole{w}{ }str}}}
\pysigstopsignatures
\end{fulllineitems}

\index{username (atributo de main.UsuarioLogin)@\spxentry{username}\spxextra{atributo de main.UsuarioLogin}}

\begin{fulllineitems}
\phantomsection\label{\detokenize{endpoints:main.UsuarioLogin.username}}
\pysigstartsignatures
\pysigline
{\sphinxbfcode{\sphinxupquote{username}}\sphinxbfcode{\sphinxupquote{\DUrole{p}{:}\DUrole{w}{ }str}}}
\pysigstopsignatures
\end{fulllineitems}


\end{fulllineitems}

\index{UsuarioRegistro (clase en main)@\spxentry{UsuarioRegistro}\spxextra{clase en main}}

\begin{fulllineitems}
\phantomsection\label{\detokenize{endpoints:main.UsuarioRegistro}}
\pysigstartsignatures
\pysiglinewithargsret
{\sphinxbfcode{\sphinxupquote{\DUrole{k}{class}\DUrole{w}{ }}}\sphinxcode{\sphinxupquote{main.}}\sphinxbfcode{\sphinxupquote{UsuarioRegistro}}}
{\sphinxparam{\DUrole{keyword-only-separator}{\DUrole{o}{\sphinxstyleabbreviation{*}}}}\sphinxparamcomma \sphinxparam{\DUrole{n}{username}\DUrole{p}{:}\DUrole{w}{ }\DUrole{n}{str}}\sphinxparamcomma \sphinxparam{\DUrole{n}{password}\DUrole{p}{:}\DUrole{w}{ }\DUrole{n}{str}}}
{}
\pysigstopsignatures
\sphinxAtStartPar
Bases: \sphinxcode{\sphinxupquote{BaseModel}}

\sphinxAtStartPar
Modelo de solicitud para registrar un nuevo usuario.
\begin{description}
\sphinxlineitem{Atributos:}
\sphinxAtStartPar
username (str): Nombre de usuario.
password (str): Contraseña en texto plano.

\end{description}
\index{model\_config (atributo de main.UsuarioRegistro)@\spxentry{model\_config}\spxextra{atributo de main.UsuarioRegistro}}

\begin{fulllineitems}
\phantomsection\label{\detokenize{endpoints:main.UsuarioRegistro.model_config}}
\pysigstartsignatures
\pysigline
{\sphinxbfcode{\sphinxupquote{model\_config}}\sphinxbfcode{\sphinxupquote{\DUrole{p}{:}\DUrole{w}{ }ClassVar\DUrole{p}{{[}}ConfigDict\DUrole{p}{{]}}}}\sphinxbfcode{\sphinxupquote{\DUrole{w}{ }\DUrole{p}{=}\DUrole{w}{ }\{\}}}}
\pysigstopsignatures
\sphinxAtStartPar
Configuration for the model, should be a dictionary conforming to {[}\sphinxtitleref{ConfigDict}{]}{[}pydantic.config.ConfigDict{]}.

\end{fulllineitems}

\index{password (atributo de main.UsuarioRegistro)@\spxentry{password}\spxextra{atributo de main.UsuarioRegistro}}

\begin{fulllineitems}
\phantomsection\label{\detokenize{endpoints:main.UsuarioRegistro.password}}
\pysigstartsignatures
\pysigline
{\sphinxbfcode{\sphinxupquote{password}}\sphinxbfcode{\sphinxupquote{\DUrole{p}{:}\DUrole{w}{ }str}}}
\pysigstopsignatures
\end{fulllineitems}

\index{username (atributo de main.UsuarioRegistro)@\spxentry{username}\spxextra{atributo de main.UsuarioRegistro}}

\begin{fulllineitems}
\phantomsection\label{\detokenize{endpoints:main.UsuarioRegistro.username}}
\pysigstartsignatures
\pysigline
{\sphinxbfcode{\sphinxupquote{username}}\sphinxbfcode{\sphinxupquote{\DUrole{p}{:}\DUrole{w}{ }str}}}
\pysigstopsignatures
\end{fulllineitems}


\end{fulllineitems}

\index{Vehiculo (clase en main)@\spxentry{Vehiculo}\spxextra{clase en main}}

\begin{fulllineitems}
\phantomsection\label{\detokenize{endpoints:main.Vehiculo}}
\pysigstartsignatures
\pysiglinewithargsret
{\sphinxbfcode{\sphinxupquote{\DUrole{k}{class}\DUrole{w}{ }}}\sphinxcode{\sphinxupquote{main.}}\sphinxbfcode{\sphinxupquote{Vehiculo}}}
{\sphinxparam{\DUrole{o}{**}\DUrole{n}{kwargs}}}
{}
\pysigstopsignatures
\sphinxAtStartPar
Bases: {\hyperref[\detokenize{modelos:main.Base}]{\sphinxcrossref{\sphinxcode{\sphinxupquote{Base}}}}}

\sphinxAtStartPar
Modelo ORM que representa un vehículo registrado.
\begin{description}
\sphinxlineitem{Atributos:}
\sphinxAtStartPar
id (int): ID del vehículo.
marca (str): Marca del vehículo.
modelo (str): Modelo del vehículo.
year (int): Año de fabricación.
rpm (int): Revoluciones por minuto.
velocidad (int): Velocidad actual.
vin (str): Número VIN único del vehículo.
revision (str): Información de revisión técnica.
usuario\_id (int): ID del usuario al que pertenece el vehículo.

\sphinxlineitem{Relaciones:}
\sphinxAtStartPar
usuario (Usuario): Usuario propietario.
errores (List{[}ErrorVehiculo{]}): Lista de errores asociados.
informes\_compartidos (List{[}InformeCompartido{]}): Informes generados con token público.

\end{description}
\index{errores (atributo de main.Vehiculo)@\spxentry{errores}\spxextra{atributo de main.Vehiculo}}

\begin{fulllineitems}
\phantomsection\label{\detokenize{endpoints:main.Vehiculo.errores}}
\pysigstartsignatures
\pysigline
{\sphinxbfcode{\sphinxupquote{errores}}}
\pysigstopsignatures
\end{fulllineitems}

\index{id (atributo de main.Vehiculo)@\spxentry{id}\spxextra{atributo de main.Vehiculo}}

\begin{fulllineitems}
\phantomsection\label{\detokenize{endpoints:main.Vehiculo.id}}
\pysigstartsignatures
\pysigline
{\sphinxbfcode{\sphinxupquote{id}}}
\pysigstopsignatures
\end{fulllineitems}

\index{informes\_compartidos (atributo de main.Vehiculo)@\spxentry{informes\_compartidos}\spxextra{atributo de main.Vehiculo}}

\begin{fulllineitems}
\phantomsection\label{\detokenize{endpoints:main.Vehiculo.informes_compartidos}}
\pysigstartsignatures
\pysigline
{\sphinxbfcode{\sphinxupquote{informes\_compartidos}}}
\pysigstopsignatures
\end{fulllineitems}

\index{marca (atributo de main.Vehiculo)@\spxentry{marca}\spxextra{atributo de main.Vehiculo}}

\begin{fulllineitems}
\phantomsection\label{\detokenize{endpoints:main.Vehiculo.marca}}
\pysigstartsignatures
\pysigline
{\sphinxbfcode{\sphinxupquote{marca}}}
\pysigstopsignatures
\end{fulllineitems}

\index{modelo (atributo de main.Vehiculo)@\spxentry{modelo}\spxextra{atributo de main.Vehiculo}}

\begin{fulllineitems}
\phantomsection\label{\detokenize{endpoints:main.Vehiculo.modelo}}
\pysigstartsignatures
\pysigline
{\sphinxbfcode{\sphinxupquote{modelo}}}
\pysigstopsignatures
\end{fulllineitems}

\index{revision (atributo de main.Vehiculo)@\spxentry{revision}\spxextra{atributo de main.Vehiculo}}

\begin{fulllineitems}
\phantomsection\label{\detokenize{endpoints:main.Vehiculo.revision}}
\pysigstartsignatures
\pysigline
{\sphinxbfcode{\sphinxupquote{revision}}}
\pysigstopsignatures
\end{fulllineitems}

\index{rpm (atributo de main.Vehiculo)@\spxentry{rpm}\spxextra{atributo de main.Vehiculo}}

\begin{fulllineitems}
\phantomsection\label{\detokenize{endpoints:main.Vehiculo.rpm}}
\pysigstartsignatures
\pysigline
{\sphinxbfcode{\sphinxupquote{rpm}}}
\pysigstopsignatures
\end{fulllineitems}

\index{usuario (atributo de main.Vehiculo)@\spxentry{usuario}\spxextra{atributo de main.Vehiculo}}

\begin{fulllineitems}
\phantomsection\label{\detokenize{endpoints:main.Vehiculo.usuario}}
\pysigstartsignatures
\pysigline
{\sphinxbfcode{\sphinxupquote{usuario}}}
\pysigstopsignatures
\end{fulllineitems}

\index{usuario\_id (atributo de main.Vehiculo)@\spxentry{usuario\_id}\spxextra{atributo de main.Vehiculo}}

\begin{fulllineitems}
\phantomsection\label{\detokenize{endpoints:main.Vehiculo.usuario_id}}
\pysigstartsignatures
\pysigline
{\sphinxbfcode{\sphinxupquote{usuario\_id}}}
\pysigstopsignatures
\end{fulllineitems}

\index{velocidad (atributo de main.Vehiculo)@\spxentry{velocidad}\spxextra{atributo de main.Vehiculo}}

\begin{fulllineitems}
\phantomsection\label{\detokenize{endpoints:main.Vehiculo.velocidad}}
\pysigstartsignatures
\pysigline
{\sphinxbfcode{\sphinxupquote{velocidad}}}
\pysigstopsignatures
\end{fulllineitems}

\index{vin (atributo de main.Vehiculo)@\spxentry{vin}\spxextra{atributo de main.Vehiculo}}

\begin{fulllineitems}
\phantomsection\label{\detokenize{endpoints:main.Vehiculo.vin}}
\pysigstartsignatures
\pysigline
{\sphinxbfcode{\sphinxupquote{vin}}}
\pysigstopsignatures
\end{fulllineitems}

\index{year (atributo de main.Vehiculo)@\spxentry{year}\spxextra{atributo de main.Vehiculo}}

\begin{fulllineitems}
\phantomsection\label{\detokenize{endpoints:main.Vehiculo.year}}
\pysigstartsignatures
\pysigline
{\sphinxbfcode{\sphinxupquote{year}}}
\pysigstopsignatures
\end{fulllineitems}


\end{fulllineitems}

\index{VehiculoEdicion (clase en main)@\spxentry{VehiculoEdicion}\spxextra{clase en main}}

\begin{fulllineitems}
\phantomsection\label{\detokenize{endpoints:main.VehiculoEdicion}}
\pysigstartsignatures
\pysiglinewithargsret
{\sphinxbfcode{\sphinxupquote{\DUrole{k}{class}\DUrole{w}{ }}}\sphinxcode{\sphinxupquote{main.}}\sphinxbfcode{\sphinxupquote{VehiculoEdicion}}}
{\sphinxparam{\DUrole{keyword-only-separator}{\DUrole{o}{\sphinxstyleabbreviation{*}}}}\sphinxparamcomma \sphinxparam{\DUrole{n}{marca}\DUrole{p}{:}\DUrole{w}{ }\DUrole{n}{str}}\sphinxparamcomma \sphinxparam{\DUrole{n}{modelo}\DUrole{p}{:}\DUrole{w}{ }\DUrole{n}{str}}\sphinxparamcomma \sphinxparam{\DUrole{n}{year}\DUrole{p}{:}\DUrole{w}{ }\DUrole{n}{int}}\sphinxparamcomma \sphinxparam{\DUrole{n}{rpm}\DUrole{p}{:}\DUrole{w}{ }\DUrole{n}{int}}\sphinxparamcomma \sphinxparam{\DUrole{n}{velocidad}\DUrole{p}{:}\DUrole{w}{ }\DUrole{n}{int}}\sphinxparamcomma \sphinxparam{\DUrole{n}{vin}\DUrole{p}{:}\DUrole{w}{ }\DUrole{n}{str}}}
{}
\pysigstopsignatures
\sphinxAtStartPar
Bases: \sphinxcode{\sphinxupquote{BaseModel}}

\sphinxAtStartPar
Modelo de solicitud para editar un vehículo existente.
\begin{description}
\sphinxlineitem{Atributos:}
\sphinxAtStartPar
marca (str): Marca del vehículo.
modelo (str): Modelo del vehículo.
year (int): Año de fabricación.
rpm (int): Revoluciones por minuto.
velocidad (int): Velocidad actual.
vin (str): Número VIN del vehículo.

\end{description}
\index{marca (atributo de main.VehiculoEdicion)@\spxentry{marca}\spxextra{atributo de main.VehiculoEdicion}}

\begin{fulllineitems}
\phantomsection\label{\detokenize{endpoints:main.VehiculoEdicion.marca}}
\pysigstartsignatures
\pysigline
{\sphinxbfcode{\sphinxupquote{marca}}\sphinxbfcode{\sphinxupquote{\DUrole{p}{:}\DUrole{w}{ }str}}}
\pysigstopsignatures
\end{fulllineitems}

\index{model\_config (atributo de main.VehiculoEdicion)@\spxentry{model\_config}\spxextra{atributo de main.VehiculoEdicion}}

\begin{fulllineitems}
\phantomsection\label{\detokenize{endpoints:main.VehiculoEdicion.model_config}}
\pysigstartsignatures
\pysigline
{\sphinxbfcode{\sphinxupquote{model\_config}}\sphinxbfcode{\sphinxupquote{\DUrole{p}{:}\DUrole{w}{ }ClassVar\DUrole{p}{{[}}ConfigDict\DUrole{p}{{]}}}}\sphinxbfcode{\sphinxupquote{\DUrole{w}{ }\DUrole{p}{=}\DUrole{w}{ }\{\}}}}
\pysigstopsignatures
\sphinxAtStartPar
Configuration for the model, should be a dictionary conforming to {[}\sphinxtitleref{ConfigDict}{]}{[}pydantic.config.ConfigDict{]}.

\end{fulllineitems}

\index{modelo (atributo de main.VehiculoEdicion)@\spxentry{modelo}\spxextra{atributo de main.VehiculoEdicion}}

\begin{fulllineitems}
\phantomsection\label{\detokenize{endpoints:main.VehiculoEdicion.modelo}}
\pysigstartsignatures
\pysigline
{\sphinxbfcode{\sphinxupquote{modelo}}\sphinxbfcode{\sphinxupquote{\DUrole{p}{:}\DUrole{w}{ }str}}}
\pysigstopsignatures
\end{fulllineitems}

\index{rpm (atributo de main.VehiculoEdicion)@\spxentry{rpm}\spxextra{atributo de main.VehiculoEdicion}}

\begin{fulllineitems}
\phantomsection\label{\detokenize{endpoints:main.VehiculoEdicion.rpm}}
\pysigstartsignatures
\pysigline
{\sphinxbfcode{\sphinxupquote{rpm}}\sphinxbfcode{\sphinxupquote{\DUrole{p}{:}\DUrole{w}{ }int}}}
\pysigstopsignatures
\end{fulllineitems}

\index{velocidad (atributo de main.VehiculoEdicion)@\spxentry{velocidad}\spxextra{atributo de main.VehiculoEdicion}}

\begin{fulllineitems}
\phantomsection\label{\detokenize{endpoints:main.VehiculoEdicion.velocidad}}
\pysigstartsignatures
\pysigline
{\sphinxbfcode{\sphinxupquote{velocidad}}\sphinxbfcode{\sphinxupquote{\DUrole{p}{:}\DUrole{w}{ }int}}}
\pysigstopsignatures
\end{fulllineitems}

\index{vin (atributo de main.VehiculoEdicion)@\spxentry{vin}\spxextra{atributo de main.VehiculoEdicion}}

\begin{fulllineitems}
\phantomsection\label{\detokenize{endpoints:main.VehiculoEdicion.vin}}
\pysigstartsignatures
\pysigline
{\sphinxbfcode{\sphinxupquote{vin}}\sphinxbfcode{\sphinxupquote{\DUrole{p}{:}\DUrole{w}{ }str}}}
\pysigstopsignatures
\end{fulllineitems}

\index{year (atributo de main.VehiculoEdicion)@\spxentry{year}\spxextra{atributo de main.VehiculoEdicion}}

\begin{fulllineitems}
\phantomsection\label{\detokenize{endpoints:main.VehiculoEdicion.year}}
\pysigstartsignatures
\pysigline
{\sphinxbfcode{\sphinxupquote{year}}\sphinxbfcode{\sphinxupquote{\DUrole{p}{:}\DUrole{w}{ }int}}}
\pysigstopsignatures
\end{fulllineitems}


\end{fulllineitems}

\index{VehiculoRegistro (clase en main)@\spxentry{VehiculoRegistro}\spxextra{clase en main}}

\begin{fulllineitems}
\phantomsection\label{\detokenize{endpoints:main.VehiculoRegistro}}
\pysigstartsignatures
\pysiglinewithargsret
{\sphinxbfcode{\sphinxupquote{\DUrole{k}{class}\DUrole{w}{ }}}\sphinxcode{\sphinxupquote{main.}}\sphinxbfcode{\sphinxupquote{VehiculoRegistro}}}
{\sphinxparam{\DUrole{keyword-only-separator}{\DUrole{o}{\sphinxstyleabbreviation{*}}}}\sphinxparamcomma \sphinxparam{\DUrole{n}{marca}\DUrole{p}{:}\DUrole{w}{ }\DUrole{n}{str}}\sphinxparamcomma \sphinxparam{\DUrole{n}{modelo}\DUrole{p}{:}\DUrole{w}{ }\DUrole{n}{str}}\sphinxparamcomma \sphinxparam{\DUrole{n}{year}\DUrole{p}{:}\DUrole{w}{ }\DUrole{n}{int}}\sphinxparamcomma \sphinxparam{\DUrole{n}{rpm}\DUrole{p}{:}\DUrole{w}{ }\DUrole{n}{int}}\sphinxparamcomma \sphinxparam{\DUrole{n}{velocidad}\DUrole{p}{:}\DUrole{w}{ }\DUrole{n}{int}}\sphinxparamcomma \sphinxparam{\DUrole{n}{vin}\DUrole{p}{:}\DUrole{w}{ }\DUrole{n}{str}}\sphinxparamcomma \sphinxparam{\DUrole{n}{revision}\DUrole{p}{:}\DUrole{w}{ }\DUrole{n}{dict}}}
{}
\pysigstopsignatures
\sphinxAtStartPar
Bases: \sphinxcode{\sphinxupquote{BaseModel}}

\sphinxAtStartPar
Modelo de solicitud para registrar un nuevo vehículo.
\begin{description}
\sphinxlineitem{Atributos:}
\sphinxAtStartPar
marca (str): Marca del vehículo.
modelo (str): Modelo del vehículo.
year (int): Año del vehículo.
rpm (int): RPM del motor.
velocidad (int): Velocidad del vehículo.
vin (str): Número VIN único del vehículo.
revision (dict): Detalles de la revisión técnica (estructura flexible).

\end{description}
\index{marca (atributo de main.VehiculoRegistro)@\spxentry{marca}\spxextra{atributo de main.VehiculoRegistro}}

\begin{fulllineitems}
\phantomsection\label{\detokenize{endpoints:main.VehiculoRegistro.marca}}
\pysigstartsignatures
\pysigline
{\sphinxbfcode{\sphinxupquote{marca}}\sphinxbfcode{\sphinxupquote{\DUrole{p}{:}\DUrole{w}{ }str}}}
\pysigstopsignatures
\end{fulllineitems}

\index{model\_config (atributo de main.VehiculoRegistro)@\spxentry{model\_config}\spxextra{atributo de main.VehiculoRegistro}}

\begin{fulllineitems}
\phantomsection\label{\detokenize{endpoints:main.VehiculoRegistro.model_config}}
\pysigstartsignatures
\pysigline
{\sphinxbfcode{\sphinxupquote{model\_config}}\sphinxbfcode{\sphinxupquote{\DUrole{p}{:}\DUrole{w}{ }ClassVar\DUrole{p}{{[}}ConfigDict\DUrole{p}{{]}}}}\sphinxbfcode{\sphinxupquote{\DUrole{w}{ }\DUrole{p}{=}\DUrole{w}{ }\{\}}}}
\pysigstopsignatures
\sphinxAtStartPar
Configuration for the model, should be a dictionary conforming to {[}\sphinxtitleref{ConfigDict}{]}{[}pydantic.config.ConfigDict{]}.

\end{fulllineitems}

\index{modelo (atributo de main.VehiculoRegistro)@\spxentry{modelo}\spxextra{atributo de main.VehiculoRegistro}}

\begin{fulllineitems}
\phantomsection\label{\detokenize{endpoints:main.VehiculoRegistro.modelo}}
\pysigstartsignatures
\pysigline
{\sphinxbfcode{\sphinxupquote{modelo}}\sphinxbfcode{\sphinxupquote{\DUrole{p}{:}\DUrole{w}{ }str}}}
\pysigstopsignatures
\end{fulllineitems}

\index{revision (atributo de main.VehiculoRegistro)@\spxentry{revision}\spxextra{atributo de main.VehiculoRegistro}}

\begin{fulllineitems}
\phantomsection\label{\detokenize{endpoints:main.VehiculoRegistro.revision}}
\pysigstartsignatures
\pysigline
{\sphinxbfcode{\sphinxupquote{revision}}\sphinxbfcode{\sphinxupquote{\DUrole{p}{:}\DUrole{w}{ }dict}}}
\pysigstopsignatures
\end{fulllineitems}

\index{rpm (atributo de main.VehiculoRegistro)@\spxentry{rpm}\spxextra{atributo de main.VehiculoRegistro}}

\begin{fulllineitems}
\phantomsection\label{\detokenize{endpoints:main.VehiculoRegistro.rpm}}
\pysigstartsignatures
\pysigline
{\sphinxbfcode{\sphinxupquote{rpm}}\sphinxbfcode{\sphinxupquote{\DUrole{p}{:}\DUrole{w}{ }int}}}
\pysigstopsignatures
\end{fulllineitems}

\index{velocidad (atributo de main.VehiculoRegistro)@\spxentry{velocidad}\spxextra{atributo de main.VehiculoRegistro}}

\begin{fulllineitems}
\phantomsection\label{\detokenize{endpoints:main.VehiculoRegistro.velocidad}}
\pysigstartsignatures
\pysigline
{\sphinxbfcode{\sphinxupquote{velocidad}}\sphinxbfcode{\sphinxupquote{\DUrole{p}{:}\DUrole{w}{ }int}}}
\pysigstopsignatures
\end{fulllineitems}

\index{vin (atributo de main.VehiculoRegistro)@\spxentry{vin}\spxextra{atributo de main.VehiculoRegistro}}

\begin{fulllineitems}
\phantomsection\label{\detokenize{endpoints:main.VehiculoRegistro.vin}}
\pysigstartsignatures
\pysigline
{\sphinxbfcode{\sphinxupquote{vin}}\sphinxbfcode{\sphinxupquote{\DUrole{p}{:}\DUrole{w}{ }str}}}
\pysigstopsignatures
\end{fulllineitems}

\index{year (atributo de main.VehiculoRegistro)@\spxentry{year}\spxextra{atributo de main.VehiculoRegistro}}

\begin{fulllineitems}
\phantomsection\label{\detokenize{endpoints:main.VehiculoRegistro.year}}
\pysigstartsignatures
\pysigline
{\sphinxbfcode{\sphinxupquote{year}}\sphinxbfcode{\sphinxupquote{\DUrole{p}{:}\DUrole{w}{ }int}}}
\pysigstopsignatures
\end{fulllineitems}


\end{fulllineitems}

\index{crear\_informe() (en el módulo main)@\spxentry{crear\_informe()}\spxextra{en el módulo main}}

\begin{fulllineitems}
\phantomsection\label{\detokenize{endpoints:main.crear_informe}}
\pysigstartsignatures
\pysiglinewithargsret
{\sphinxbfcode{\sphinxupquote{\DUrole{k}{async}\DUrole{w}{ }}}\sphinxcode{\sphinxupquote{main.}}\sphinxbfcode{\sphinxupquote{crear\_informe}}}
{\sphinxparam{\DUrole{n}{vehiculo\_id}\DUrole{p}{:}\DUrole{w}{ }\DUrole{n}{int}}\sphinxparamcomma \sphinxparam{\DUrole{n}{request}\DUrole{p}{:}\DUrole{w}{ }\DUrole{n}{{\hyperref[\detokenize{modelos:main.InformeRequest}]{\sphinxcrossref{InformeRequest}}}}}\sphinxparamcomma \sphinxparam{\DUrole{n}{usuario}\DUrole{p}{:}\DUrole{w}{ }\DUrole{n}{{\hyperref[\detokenize{modelos:main.Usuario}]{\sphinxcrossref{Usuario}}}}\DUrole{w}{ }\DUrole{o}{=}\DUrole{w}{ }\DUrole{default_value}{Depends(obtener\_usuario\_desde\_token)}}\sphinxparamcomma \sphinxparam{\DUrole{n}{db}\DUrole{p}{:}\DUrole{w}{ }\DUrole{n}{Session}\DUrole{w}{ }\DUrole{o}{=}\DUrole{w}{ }\DUrole{default_value}{Depends(get\_db)}}}
{}
\pysigstopsignatures
\sphinxAtStartPar
Crea un informe de errores del vehículo y lo envía al email del cliente.

\sphinxAtStartPar
Este endpoint genera un enlace único que da acceso a una vista del informe de diagnóstico del vehículo. Se envía un correo al cliente con dicho enlace.
\begin{quote}\begin{description}
\sphinxlineitem{Parámetros}\begin{itemize}
\item {} 
\sphinxAtStartPar
\sphinxstyleliteralstrong{\sphinxupquote{vehiculo\_id}} (\sphinxstyleliteralemphasis{\sphinxupquote{int}}) \textendash{} ID del vehículo del que se desea generar el informe.

\item {} 
\sphinxAtStartPar
\sphinxstyleliteralstrong{\sphinxupquote{request}} ({\hyperref[\detokenize{modelos:main.InformeRequest}]{\sphinxcrossref{\sphinxstyleliteralemphasis{\sphinxupquote{InformeRequest}}}}}) \textendash{} Objeto que contiene el email del cliente.

\item {} 
\sphinxAtStartPar
\sphinxstyleliteralstrong{\sphinxupquote{usuario}} ({\hyperref[\detokenize{modelos:main.Usuario}]{\sphinxcrossref{\sphinxstyleliteralemphasis{\sphinxupquote{Usuario}}}}}) \textendash{} Usuario autenticado mediante JWT.

\item {} 
\sphinxAtStartPar
\sphinxstyleliteralstrong{\sphinxupquote{db}} (\sphinxstyleliteralemphasis{\sphinxupquote{Session}}) \textendash{} Sesión activa de la base de datos.

\end{itemize}

\sphinxlineitem{Devuelve}
\sphinxAtStartPar
Mensaje de éxito, token generado y enlace de acceso.

\sphinxlineitem{Tipo del valor devuelto}
\sphinxAtStartPar
dict

\sphinxlineitem{Muestra}\begin{itemize}
\item {} 
\sphinxAtStartPar
\sphinxstyleliteralstrong{\sphinxupquote{HTTPException 400}} \textendash{} Si el email no es válido.

\item {} 
\sphinxAtStartPar
\sphinxstyleliteralstrong{\sphinxupquote{HTTPException 404}} \textendash{} Si el vehículo no pertenece al usuario.

\item {} 
\sphinxAtStartPar
\sphinxstyleliteralstrong{\sphinxupquote{HTTPException 500}} \textendash{} Si ocurre un error al guardar el informe o enviar el correo.

\end{itemize}

\end{description}\end{quote}

\end{fulllineitems}

\index{crear\_token() (en el módulo main)@\spxentry{crear\_token()}\spxextra{en el módulo main}}

\begin{fulllineitems}
\phantomsection\label{\detokenize{endpoints:main.crear_token}}
\pysigstartsignatures
\pysiglinewithargsret
{\sphinxcode{\sphinxupquote{main.}}\sphinxbfcode{\sphinxupquote{crear\_token}}}
{\sphinxparam{\DUrole{n}{data}\DUrole{p}{:}\DUrole{w}{ }\DUrole{n}{dict}}\sphinxparamcomma \sphinxparam{\DUrole{n}{expira\_en}\DUrole{p}{:}\DUrole{w}{ }\DUrole{n}{int}\DUrole{w}{ }\DUrole{o}{=}\DUrole{w}{ }\DUrole{default_value}{300}}}
{}
\pysigstopsignatures
\sphinxAtStartPar
Genera un token JWT con los datos proporcionados y un tiempo de expiración opcional.
\begin{quote}\begin{description}
\sphinxlineitem{Parámetros}\begin{itemize}
\item {} 
\sphinxAtStartPar
\sphinxstyleliteralstrong{\sphinxupquote{datos}} (\sphinxstyleliteralemphasis{\sphinxupquote{dict}}) \textendash{} Datos a incluir en el payload del token.

\item {} 
\sphinxAtStartPar
\sphinxstyleliteralstrong{\sphinxupquote{tiempo\_expiracion}} (\sphinxstyleliteralemphasis{\sphinxupquote{Optional}}\sphinxstyleliteralemphasis{\sphinxupquote{{[}}}\sphinxstyleliteralemphasis{\sphinxupquote{timedelta}}\sphinxstyleliteralemphasis{\sphinxupquote{{]}}}) \textendash{} Tiempo personalizado de expiración. Si no se especifica, se usarán 30 minutos por defecto.

\end{itemize}

\sphinxlineitem{Devuelve}
\sphinxAtStartPar
Token JWT firmado.

\sphinxlineitem{Tipo del valor devuelto}
\sphinxAtStartPar
str

\sphinxlineitem{Muestra}
\sphinxAtStartPar
\sphinxstyleliteralstrong{\sphinxupquote{Exception}} \textendash{} Si hay un error al codificar el token.

\end{description}\end{quote}

\end{fulllineitems}

\index{editar\_vehiculo() (en el módulo main)@\spxentry{editar\_vehiculo()}\spxextra{en el módulo main}}

\begin{fulllineitems}
\phantomsection\label{\detokenize{endpoints:main.editar_vehiculo}}
\pysigstartsignatures
\pysiglinewithargsret
{\sphinxcode{\sphinxupquote{main.}}\sphinxbfcode{\sphinxupquote{editar\_vehiculo}}}
{\sphinxparam{\DUrole{n}{vehiculo\_id}\DUrole{p}{:}\DUrole{w}{ }\DUrole{n}{int}}\sphinxparamcomma \sphinxparam{\DUrole{n}{datos}\DUrole{p}{:}\DUrole{w}{ }\DUrole{n}{{\hyperref[\detokenize{modelos:main.VehiculoEdicion}]{\sphinxcrossref{VehiculoEdicion}}}}}\sphinxparamcomma \sphinxparam{\DUrole{n}{usuario}\DUrole{p}{:}\DUrole{w}{ }\DUrole{n}{{\hyperref[\detokenize{modelos:main.Usuario}]{\sphinxcrossref{Usuario}}}}\DUrole{w}{ }\DUrole{o}{=}\DUrole{w}{ }\DUrole{default_value}{Depends(obtener\_usuario\_desde\_token)}}\sphinxparamcomma \sphinxparam{\DUrole{n}{db}\DUrole{p}{:}\DUrole{w}{ }\DUrole{n}{Session}\DUrole{w}{ }\DUrole{o}{=}\DUrole{w}{ }\DUrole{default_value}{Depends(get\_db)}}}
{}
\pysigstopsignatures
\sphinxAtStartPar
Actualiza los datos de un vehículo existente del usuario autenticado.
\begin{quote}\begin{description}
\sphinxlineitem{Parámetros}\begin{itemize}
\item {} 
\sphinxAtStartPar
\sphinxstyleliteralstrong{\sphinxupquote{vehiculo\_id}} (\sphinxstyleliteralemphasis{\sphinxupquote{int}}) \textendash{} ID del vehículo a modificar.

\item {} 
\sphinxAtStartPar
\sphinxstyleliteralstrong{\sphinxupquote{datos\_actualizados}} (\sphinxstyleliteralemphasis{\sphinxupquote{VehiculoBase}}) \textendash{} Nuevos datos del vehículo.

\item {} 
\sphinxAtStartPar
\sphinxstyleliteralstrong{\sphinxupquote{db}} (\sphinxstyleliteralemphasis{\sphinxupquote{Session}}) \textendash{} Sesión de base de datos.

\item {} 
\sphinxAtStartPar
\sphinxstyleliteralstrong{\sphinxupquote{usuario}} ({\hyperref[\detokenize{modelos:main.Usuario}]{\sphinxcrossref{\sphinxstyleliteralemphasis{\sphinxupquote{Usuario}}}}}) \textendash{} Usuario autenticado.

\end{itemize}

\sphinxlineitem{Devuelve}
\sphinxAtStartPar
Mensaje de éxito.

\sphinxlineitem{Tipo del valor devuelto}
\sphinxAtStartPar
dict

\sphinxlineitem{Muestra}
\sphinxAtStartPar
\sphinxstyleliteralstrong{\sphinxupquote{HTTPException 404}} \textendash{} Si el vehículo no existe o no pertenece al usuario.

\end{description}\end{quote}

\end{fulllineitems}

\index{eliminar\_vehiculo() (en el módulo main)@\spxentry{eliminar\_vehiculo()}\spxextra{en el módulo main}}

\begin{fulllineitems}
\phantomsection\label{\detokenize{endpoints:main.eliminar_vehiculo}}
\pysigstartsignatures
\pysiglinewithargsret
{\sphinxcode{\sphinxupquote{main.}}\sphinxbfcode{\sphinxupquote{eliminar\_vehiculo}}}
{\sphinxparam{\DUrole{n}{vehiculo\_id}\DUrole{p}{:}\DUrole{w}{ }\DUrole{n}{int}}\sphinxparamcomma \sphinxparam{\DUrole{n}{usuario}\DUrole{p}{:}\DUrole{w}{ }\DUrole{n}{{\hyperref[\detokenize{modelos:main.Usuario}]{\sphinxcrossref{Usuario}}}}\DUrole{w}{ }\DUrole{o}{=}\DUrole{w}{ }\DUrole{default_value}{Depends(obtener\_usuario\_desde\_token)}}\sphinxparamcomma \sphinxparam{\DUrole{n}{db}\DUrole{p}{:}\DUrole{w}{ }\DUrole{n}{Session}\DUrole{w}{ }\DUrole{o}{=}\DUrole{w}{ }\DUrole{default_value}{Depends(get\_db)}}}
{}
\pysigstopsignatures
\sphinxAtStartPar
Elimina un vehículo registrado por el usuario autenticado.
\begin{quote}\begin{description}
\sphinxlineitem{Parámetros}\begin{itemize}
\item {} 
\sphinxAtStartPar
\sphinxstyleliteralstrong{\sphinxupquote{vehiculo\_id}} (\sphinxstyleliteralemphasis{\sphinxupquote{int}}) \textendash{} ID del vehículo a eliminar.

\item {} 
\sphinxAtStartPar
\sphinxstyleliteralstrong{\sphinxupquote{db}} (\sphinxstyleliteralemphasis{\sphinxupquote{Session}}) \textendash{} Sesión de base de datos.

\item {} 
\sphinxAtStartPar
\sphinxstyleliteralstrong{\sphinxupquote{usuario}} ({\hyperref[\detokenize{modelos:main.Usuario}]{\sphinxcrossref{\sphinxstyleliteralemphasis{\sphinxupquote{Usuario}}}}}) \textendash{} Usuario autenticado mediante JWT.

\end{itemize}

\sphinxlineitem{Devuelve}
\sphinxAtStartPar
Mensaje de éxito.

\sphinxlineitem{Tipo del valor devuelto}
\sphinxAtStartPar
dict

\sphinxlineitem{Muestra}
\sphinxAtStartPar
\sphinxstyleliteralstrong{\sphinxupquote{HTTPException 404}} \textendash{} Si el vehículo no existe o no pertenece al usuario.

\end{description}\end{quote}

\end{fulllineitems}

\index{fm (en el módulo main)@\spxentry{fm}\spxextra{en el módulo main}}

\begin{fulllineitems}
\phantomsection\label{\detokenize{endpoints:main.fm}}
\pysigstartsignatures
\pysigline
{\sphinxcode{\sphinxupquote{main.}}\sphinxbfcode{\sphinxupquote{fm}}\sphinxbfcode{\sphinxupquote{\DUrole{w}{ }\DUrole{p}{=}\DUrole{w}{ }\textless{}fastapi\_mail.fastmail.FastMail object\textgreater{}}}}
\pysigstopsignatures
\sphinxAtStartPar
Configuración de seguridad:
\begin{itemize}
\item {} 
\sphinxAtStartPar
\sphinxtitleref{SECRET\_KEY}, \sphinxtitleref{ALGORITHM} y tiempo de expiración definen la seguridad del JWT.

\item {} 
\sphinxAtStartPar
\sphinxtitleref{pwd\_context} se usa para hashear contraseñas con bcrypt.

\item {} 
\sphinxAtStartPar
\sphinxtitleref{oauth2\_scheme} se usa como dependencia para extraer el token del header Authorization.

\end{itemize}

\end{fulllineitems}

\index{get\_car\_image() (en el módulo main)@\spxentry{get\_car\_image()}\spxextra{en el módulo main}}

\begin{fulllineitems}
\phantomsection\label{\detokenize{endpoints:main.get_car_image}}
\pysigstartsignatures
\pysiglinewithargsret
{\sphinxcode{\sphinxupquote{main.}}\sphinxbfcode{\sphinxupquote{get\_car\_image}}}
{\sphinxparam{\DUrole{n}{searchTerm}\DUrole{p}{:}\DUrole{w}{ }\DUrole{n}{str}}}
{}
\pysigstopsignatures
\sphinxAtStartPar
Obtiene una URL de imagen representativa de un vehículo usando el término de búsqueda proporcionado.

\sphinxAtStartPar
Este endpoint consulta la API externa de carimagery.com para devolver la URL de una imagen que coincida con el término (por ejemplo, «Toyota Corolla 2020»).
\begin{quote}\begin{description}
\sphinxlineitem{Parámetros}
\sphinxAtStartPar
\sphinxstyleliteralstrong{\sphinxupquote{searchTerm}} (\sphinxstyleliteralemphasis{\sphinxupquote{str}}) \textendash{} Término de búsqueda del vehículo (marca, modelo, año, etc.).

\sphinxlineitem{Devuelve}
\sphinxAtStartPar
URL de la imagen del vehículo.

\sphinxlineitem{Tipo del valor devuelto}
\sphinxAtStartPar
str

\sphinxlineitem{Muestra}
\sphinxAtStartPar
\sphinxstyleliteralstrong{\sphinxupquote{HTTPException 500}} \textendash{} Si hay un error al consultar la API externa.

\end{description}\end{quote}

\end{fulllineitems}

\index{get\_db() (en el módulo main)@\spxentry{get\_db()}\spxextra{en el módulo main}}

\begin{fulllineitems}
\phantomsection\label{\detokenize{endpoints:main.get_db}}
\pysigstartsignatures
\pysiglinewithargsret
{\sphinxcode{\sphinxupquote{main.}}\sphinxbfcode{\sphinxupquote{get\_db}}}
{}
{}
\pysigstopsignatures
\sphinxAtStartPar
Dependencia de FastAPI para obtener una sesión de base de datos.

\sphinxAtStartPar
Se utiliza con \sphinxtitleref{Depends(get\_db)} para abrir una sesión, cederla al endpoint y cerrarla automáticamente.

\end{fulllineitems}

\index{guardar\_errores() (en el módulo main)@\spxentry{guardar\_errores()}\spxextra{en el módulo main}}

\begin{fulllineitems}
\phantomsection\label{\detokenize{endpoints:main.guardar_errores}}
\pysigstartsignatures
\pysiglinewithargsret
{\sphinxcode{\sphinxupquote{main.}}\sphinxbfcode{\sphinxupquote{guardar\_errores}}}
{\sphinxparam{\DUrole{n}{datos}\DUrole{p}{:}\DUrole{w}{ }\DUrole{n}{{\hyperref[\detokenize{modelos:main.ErrorVehiculoRegistro}]{\sphinxcrossref{ErrorVehiculoRegistro}}}}}\sphinxparamcomma \sphinxparam{\DUrole{n}{usuario}\DUrole{p}{:}\DUrole{w}{ }\DUrole{n}{{\hyperref[\detokenize{modelos:main.Usuario}]{\sphinxcrossref{Usuario}}}}\DUrole{w}{ }\DUrole{o}{=}\DUrole{w}{ }\DUrole{default_value}{Depends(obtener\_usuario\_desde\_token)}}\sphinxparamcomma \sphinxparam{\DUrole{n}{db}\DUrole{p}{:}\DUrole{w}{ }\DUrole{n}{Session}\DUrole{w}{ }\DUrole{o}{=}\DUrole{w}{ }\DUrole{default_value}{Depends(get\_db)}}}
{}
\pysigstopsignatures
\sphinxAtStartPar
Guarda una lista de códigos de error OBD\sphinxhyphen{}II (DTC) asociados a un vehículo del usuario autenticado.

\sphinxAtStartPar
Este endpoint es utilizado por el cliente Python que recibe errores del escáner OBD\sphinxhyphen{}II y los envía al backend para su almacenamiento.
\begin{quote}\begin{description}
\sphinxlineitem{Parámetros}\begin{itemize}
\item {} 
\sphinxAtStartPar
\sphinxstyleliteralstrong{\sphinxupquote{datos}} ({\hyperref[\detokenize{modelos:main.ErrorVehiculoRegistro}]{\sphinxcrossref{\sphinxstyleliteralemphasis{\sphinxupquote{ErrorVehiculoRegistro}}}}}) \textendash{} Objeto que contiene el ID del vehículo y una lista de códigos DTC.

\item {} 
\sphinxAtStartPar
\sphinxstyleliteralstrong{\sphinxupquote{usuario}} ({\hyperref[\detokenize{modelos:main.Usuario}]{\sphinxcrossref{\sphinxstyleliteralemphasis{\sphinxupquote{Usuario}}}}}) \textendash{} Usuario autenticado, obtenido desde el token JWT.

\item {} 
\sphinxAtStartPar
\sphinxstyleliteralstrong{\sphinxupquote{db}} (\sphinxstyleliteralemphasis{\sphinxupquote{Session}}) \textendash{} Sesión activa de la base de datos.

\end{itemize}

\sphinxlineitem{Devuelve}
\sphinxAtStartPar
Mensaje de confirmación si los errores fueron guardados correctamente.

\sphinxlineitem{Tipo del valor devuelto}
\sphinxAtStartPar
dict

\sphinxlineitem{Muestra}\begin{itemize}
\item {} 
\sphinxAtStartPar
\sphinxstyleliteralstrong{\sphinxupquote{HTTPException 400}} \textendash{} \begin{itemize}
\item {} 
\sphinxAtStartPar
Si el ID del vehículo no es válido (no entero o negativo).
    \sphinxhyphen{} Si la lista de códigos está vacía o contiene valores vacíos.
    \sphinxhyphen{} Si hay códigos DTC duplicados.

\end{itemize}


\item {} 
\sphinxAtStartPar
\sphinxstyleliteralstrong{\sphinxupquote{HTTPException 404}} \textendash{} Si el vehículo no pertenece al usuario autenticado.

\item {} 
\sphinxAtStartPar
\sphinxstyleliteralstrong{\sphinxupquote{HTTPException 500}} \textendash{} Si ocurre un error inesperado al guardar en la base de datos.

\end{itemize}

\end{description}\end{quote}

\end{fulllineitems}

\index{guardar\_vehiculo() (en el módulo main)@\spxentry{guardar\_vehiculo()}\spxextra{en el módulo main}}

\begin{fulllineitems}
\phantomsection\label{\detokenize{endpoints:main.guardar_vehiculo}}
\pysigstartsignatures
\pysiglinewithargsret
{\sphinxcode{\sphinxupquote{main.}}\sphinxbfcode{\sphinxupquote{guardar\_vehiculo}}}
{\sphinxparam{\DUrole{n}{datos}\DUrole{p}{:}\DUrole{w}{ }\DUrole{n}{{\hyperref[\detokenize{modelos:main.VehiculoRegistro}]{\sphinxcrossref{VehiculoRegistro}}}}}\sphinxparamcomma \sphinxparam{\DUrole{n}{usuario}\DUrole{p}{:}\DUrole{w}{ }\DUrole{n}{{\hyperref[\detokenize{modelos:main.Usuario}]{\sphinxcrossref{Usuario}}}}\DUrole{w}{ }\DUrole{o}{=}\DUrole{w}{ }\DUrole{default_value}{Depends(obtener\_usuario\_desde\_token)}}\sphinxparamcomma \sphinxparam{\DUrole{n}{db}\DUrole{p}{:}\DUrole{w}{ }\DUrole{n}{Session}\DUrole{w}{ }\DUrole{o}{=}\DUrole{w}{ }\DUrole{default_value}{Depends(get\_db)}}}
{}
\pysigstopsignatures
\sphinxAtStartPar
Guarda un nuevo vehículo en la base de datos asociado al usuario autenticado.
\begin{quote}\begin{description}
\sphinxlineitem{Parámetros}\begin{itemize}
\item {} 
\sphinxAtStartPar
\sphinxstyleliteralstrong{\sphinxupquote{vehiculo}} (\sphinxstyleliteralemphasis{\sphinxupquote{VehiculoBase}}) \textendash{} Datos del vehículo (marca, modelo, año, color, etc.).

\item {} 
\sphinxAtStartPar
\sphinxstyleliteralstrong{\sphinxupquote{db}} (\sphinxstyleliteralemphasis{\sphinxupquote{Session}}) \textendash{} Sesión de base de datos.

\item {} 
\sphinxAtStartPar
\sphinxstyleliteralstrong{\sphinxupquote{usuario}} ({\hyperref[\detokenize{modelos:main.Usuario}]{\sphinxcrossref{\sphinxstyleliteralemphasis{\sphinxupquote{Usuario}}}}}) \textendash{} Usuario autenticado, extraído desde el token JWT.

\end{itemize}

\sphinxlineitem{Devuelve}
\sphinxAtStartPar
Mensaje de confirmación.

\sphinxlineitem{Tipo del valor devuelto}
\sphinxAtStartPar
dict

\sphinxlineitem{Muestra}
\sphinxAtStartPar
\sphinxstyleliteralstrong{\sphinxupquote{HTTPException 401}} \textendash{} Si no se proporciona un token válido.

\end{description}\end{quote}

\end{fulllineitems}

\index{login() (en el módulo main)@\spxentry{login()}\spxextra{en el módulo main}}

\begin{fulllineitems}
\phantomsection\label{\detokenize{endpoints:main.login}}
\pysigstartsignatures
\pysiglinewithargsret
{\sphinxcode{\sphinxupquote{main.}}\sphinxbfcode{\sphinxupquote{login}}}
{\sphinxparam{\DUrole{n}{datos}\DUrole{p}{:}\DUrole{w}{ }\DUrole{n}{{\hyperref[\detokenize{modelos:main.UsuarioLogin}]{\sphinxcrossref{UsuarioLogin}}}}}\sphinxparamcomma \sphinxparam{\DUrole{n}{db}\DUrole{p}{:}\DUrole{w}{ }\DUrole{n}{Session}\DUrole{w}{ }\DUrole{o}{=}\DUrole{w}{ }\DUrole{default_value}{Depends(get\_db)}}}
{}
\pysigstopsignatures
\sphinxAtStartPar
Autentica al usuario y devuelve un token JWT válido.
\begin{quote}\begin{description}
\sphinxlineitem{Parámetros}\begin{itemize}
\item {} 
\sphinxAtStartPar
\sphinxstyleliteralstrong{\sphinxupquote{datos}} ({\hyperref[\detokenize{modelos:main.UsuarioLogin}]{\sphinxcrossref{\sphinxstyleliteralemphasis{\sphinxupquote{UsuarioLogin}}}}}) \textendash{} Credenciales de usuario.

\item {} 
\sphinxAtStartPar
\sphinxstyleliteralstrong{\sphinxupquote{db}} (\sphinxstyleliteralemphasis{\sphinxupquote{Session}}) \textendash{} Sesión activa de la base de datos.

\end{itemize}

\sphinxlineitem{Devuelve}
\sphinxAtStartPar
Token JWT si la autenticación fue exitosa.

\sphinxlineitem{Tipo del valor devuelto}
\sphinxAtStartPar
dict

\sphinxlineitem{Muestra}\begin{itemize}
\item {} 
\sphinxAtStartPar
\sphinxstyleliteralstrong{\sphinxupquote{HTTPException 400}} \textendash{} Datos inválidos.

\item {} 
\sphinxAtStartPar
\sphinxstyleliteralstrong{\sphinxupquote{HTTPException 401}} \textendash{} Usuario no encontrado o contraseña incorrecta.

\item {} 
\sphinxAtStartPar
\sphinxstyleliteralstrong{\sphinxupquote{HTTPException 500}} \textendash{} Error al generar el token.

\end{itemize}

\end{description}\end{quote}

\end{fulllineitems}

\index{obtener\_errores() (en el módulo main)@\spxentry{obtener\_errores()}\spxextra{en el módulo main}}

\begin{fulllineitems}
\phantomsection\label{\detokenize{endpoints:main.obtener_errores}}
\pysigstartsignatures
\pysiglinewithargsret
{\sphinxcode{\sphinxupquote{main.}}\sphinxbfcode{\sphinxupquote{obtener\_errores}}}
{\sphinxparam{\DUrole{n}{vehiculo\_id}\DUrole{p}{:}\DUrole{w}{ }\DUrole{n}{int}}\sphinxparamcomma \sphinxparam{\DUrole{n}{usuario}\DUrole{p}{:}\DUrole{w}{ }\DUrole{n}{{\hyperref[\detokenize{modelos:main.Usuario}]{\sphinxcrossref{Usuario}}}}\DUrole{w}{ }\DUrole{o}{=}\DUrole{w}{ }\DUrole{default_value}{Depends(obtener\_usuario\_desde\_token)}}\sphinxparamcomma \sphinxparam{\DUrole{n}{db}\DUrole{p}{:}\DUrole{w}{ }\DUrole{n}{Session}\DUrole{w}{ }\DUrole{o}{=}\DUrole{w}{ }\DUrole{default_value}{Depends(get\_db)}}}
{}
\pysigstopsignatures
\sphinxAtStartPar
Devuelve todos los errores DTC (códigos OBD\sphinxhyphen{}II) asociados a un vehículo del usuario autenticado.
\begin{quote}\begin{description}
\sphinxlineitem{Parámetros}\begin{itemize}
\item {} 
\sphinxAtStartPar
\sphinxstyleliteralstrong{\sphinxupquote{vehiculo\_id}} (\sphinxstyleliteralemphasis{\sphinxupquote{int}}) \textendash{} ID del vehículo para el que se desean consultar los errores.

\item {} 
\sphinxAtStartPar
\sphinxstyleliteralstrong{\sphinxupquote{usuario}} ({\hyperref[\detokenize{modelos:main.Usuario}]{\sphinxcrossref{\sphinxstyleliteralemphasis{\sphinxupquote{Usuario}}}}}) \textendash{} Usuario autenticado mediante JWT.

\item {} 
\sphinxAtStartPar
\sphinxstyleliteralstrong{\sphinxupquote{db}} (\sphinxstyleliteralemphasis{\sphinxupquote{Session}}) \textendash{} Sesión activa de la base de datos.

\end{itemize}

\sphinxlineitem{Devuelve}
\sphinxAtStartPar
Lista de errores registrados.

\sphinxlineitem{Tipo del valor devuelto}
\sphinxAtStartPar
List{[}{\hyperref[\detokenize{modelos:main.ErrorVehiculo}]{\sphinxcrossref{ErrorVehiculo}}}{]}

\sphinxlineitem{Muestra}
\sphinxAtStartPar
\sphinxstyleliteralstrong{\sphinxupquote{HTTPException 404}} \textendash{} Si no existen errores para ese vehículo.

\end{description}\end{quote}

\end{fulllineitems}

\index{obtener\_usuario\_desde\_token() (en el módulo main)@\spxentry{obtener\_usuario\_desde\_token()}\spxextra{en el módulo main}}

\begin{fulllineitems}
\phantomsection\label{\detokenize{endpoints:main.obtener_usuario_desde_token}}
\pysigstartsignatures
\pysiglinewithargsret
{\sphinxcode{\sphinxupquote{main.}}\sphinxbfcode{\sphinxupquote{obtener\_usuario\_desde\_token}}}
{\sphinxparam{\DUrole{n}{token}\DUrole{p}{:}\DUrole{w}{ }\DUrole{n}{str}\DUrole{w}{ }\DUrole{o}{=}\DUrole{w}{ }\DUrole{default_value}{Depends(OAuth2PasswordBearer)}}\sphinxparamcomma \sphinxparam{\DUrole{n}{db}\DUrole{p}{:}\DUrole{w}{ }\DUrole{n}{Session}\DUrole{w}{ }\DUrole{o}{=}\DUrole{w}{ }\DUrole{default_value}{Depends(get\_db)}}}
{}
\pysigstopsignatures
\sphinxAtStartPar
Extrae y valida el usuario actual a partir del token JWT proporcionado.
\begin{quote}\begin{description}
\sphinxlineitem{Parámetros}\begin{itemize}
\item {} 
\sphinxAtStartPar
\sphinxstyleliteralstrong{\sphinxupquote{token}} (\sphinxstyleliteralemphasis{\sphinxupquote{str}}) \textendash{} Token JWT incluido en el encabezado de autorización.

\item {} 
\sphinxAtStartPar
\sphinxstyleliteralstrong{\sphinxupquote{db}} (\sphinxstyleliteralemphasis{\sphinxupquote{Session}}) \textendash{} Sesión de base de datos.

\end{itemize}

\sphinxlineitem{Devuelve}
\sphinxAtStartPar
Instancia del usuario autenticado.

\sphinxlineitem{Tipo del valor devuelto}
\sphinxAtStartPar
{\hyperref[\detokenize{modelos:main.Usuario}]{\sphinxcrossref{Usuario}}}

\sphinxlineitem{Muestra}
\sphinxAtStartPar
\sphinxstyleliteralstrong{\sphinxupquote{HTTPException 401}} \textendash{} Si el token es inválido o ha expirado.

\end{description}\end{quote}

\end{fulllineitems}

\index{obtener\_vehiculo() (en el módulo main)@\spxentry{obtener\_vehiculo()}\spxextra{en el módulo main}}

\begin{fulllineitems}
\phantomsection\label{\detokenize{endpoints:main.obtener_vehiculo}}
\pysigstartsignatures
\pysiglinewithargsret
{\sphinxcode{\sphinxupquote{main.}}\sphinxbfcode{\sphinxupquote{obtener\_vehiculo}}}
{\sphinxparam{\DUrole{n}{vehiculo\_id}\DUrole{p}{:}\DUrole{w}{ }\DUrole{n}{int}}\sphinxparamcomma \sphinxparam{\DUrole{n}{usuario}\DUrole{p}{:}\DUrole{w}{ }\DUrole{n}{{\hyperref[\detokenize{modelos:main.Usuario}]{\sphinxcrossref{Usuario}}}}\DUrole{w}{ }\DUrole{o}{=}\DUrole{w}{ }\DUrole{default_value}{Depends(obtener\_usuario\_desde\_token)}}\sphinxparamcomma \sphinxparam{\DUrole{n}{db}\DUrole{p}{:}\DUrole{w}{ }\DUrole{n}{Session}\DUrole{w}{ }\DUrole{o}{=}\DUrole{w}{ }\DUrole{default_value}{Depends(get\_db)}}}
{}
\pysigstopsignatures
\sphinxAtStartPar
Recupera la información de un vehículo específico registrado por el usuario autenticado.
\begin{quote}\begin{description}
\sphinxlineitem{Parámetros}\begin{itemize}
\item {} 
\sphinxAtStartPar
\sphinxstyleliteralstrong{\sphinxupquote{vehiculo\_id}} (\sphinxstyleliteralemphasis{\sphinxupquote{int}}) \textendash{} ID del vehículo a consultar.

\item {} 
\sphinxAtStartPar
\sphinxstyleliteralstrong{\sphinxupquote{usuario}} ({\hyperref[\detokenize{modelos:main.Usuario}]{\sphinxcrossref{\sphinxstyleliteralemphasis{\sphinxupquote{Usuario}}}}}) \textendash{} Usuario autenticado mediante JWT.

\item {} 
\sphinxAtStartPar
\sphinxstyleliteralstrong{\sphinxupquote{db}} (\sphinxstyleliteralemphasis{\sphinxupquote{Session}}) \textendash{} Sesión activa de la base de datos.

\end{itemize}

\sphinxlineitem{Devuelve}
\sphinxAtStartPar
Objeto del vehículo solicitado.

\sphinxlineitem{Tipo del valor devuelto}
\sphinxAtStartPar
{\hyperref[\detokenize{modelos:main.Vehiculo}]{\sphinxcrossref{Vehiculo}}}

\sphinxlineitem{Muestra}
\sphinxAtStartPar
\sphinxstyleliteralstrong{\sphinxupquote{HTTPException 404}} \textendash{} Si el vehículo no pertenece al usuario o no existe.

\end{description}\end{quote}

\end{fulllineitems}

\index{obtener\_vehiculos() (en el módulo main)@\spxentry{obtener\_vehiculos()}\spxextra{en el módulo main}}

\begin{fulllineitems}
\phantomsection\label{\detokenize{endpoints:main.obtener_vehiculos}}
\pysigstartsignatures
\pysiglinewithargsret
{\sphinxcode{\sphinxupquote{main.}}\sphinxbfcode{\sphinxupquote{obtener\_vehiculos}}}
{\sphinxparam{\DUrole{n}{usuario}\DUrole{p}{:}\DUrole{w}{ }\DUrole{n}{{\hyperref[\detokenize{modelos:main.Usuario}]{\sphinxcrossref{Usuario}}}}\DUrole{w}{ }\DUrole{o}{=}\DUrole{w}{ }\DUrole{default_value}{Depends(obtener\_usuario\_desde\_token)}}\sphinxparamcomma \sphinxparam{\DUrole{n}{db}\DUrole{p}{:}\DUrole{w}{ }\DUrole{n}{Session}\DUrole{w}{ }\DUrole{o}{=}\DUrole{w}{ }\DUrole{default_value}{Depends(get\_db)}}}
{}
\pysigstopsignatures
\sphinxAtStartPar
Obtiene todos los vehículos registrados por el usuario autenticado.
\begin{quote}\begin{description}
\sphinxlineitem{Parámetros}\begin{itemize}
\item {} 
\sphinxAtStartPar
\sphinxstyleliteralstrong{\sphinxupquote{db}} (\sphinxstyleliteralemphasis{\sphinxupquote{Session}}) \textendash{} Sesión de base de datos.

\item {} 
\sphinxAtStartPar
\sphinxstyleliteralstrong{\sphinxupquote{usuario}} ({\hyperref[\detokenize{modelos:main.Usuario}]{\sphinxcrossref{\sphinxstyleliteralemphasis{\sphinxupquote{Usuario}}}}}) \textendash{} Usuario autenticado mediante JWT.

\end{itemize}

\sphinxlineitem{Devuelve}
\sphinxAtStartPar
Lista de vehículos asociados al usuario.

\sphinxlineitem{Tipo del valor devuelto}
\sphinxAtStartPar
List{[}VehiculoBase{]}

\end{description}\end{quote}

\end{fulllineitems}

\index{register() (en el módulo main)@\spxentry{register()}\spxextra{en el módulo main}}

\begin{fulllineitems}
\phantomsection\label{\detokenize{endpoints:main.register}}
\pysigstartsignatures
\pysiglinewithargsret
{\sphinxcode{\sphinxupquote{main.}}\sphinxbfcode{\sphinxupquote{register}}}
{\sphinxparam{\DUrole{n}{datos}\DUrole{p}{:}\DUrole{w}{ }\DUrole{n}{{\hyperref[\detokenize{modelos:main.UsuarioRegistro}]{\sphinxcrossref{UsuarioRegistro}}}}}\sphinxparamcomma \sphinxparam{\DUrole{n}{db}\DUrole{p}{:}\DUrole{w}{ }\DUrole{n}{Session}\DUrole{w}{ }\DUrole{o}{=}\DUrole{w}{ }\DUrole{default_value}{Depends(get\_db)}}}
{}
\pysigstopsignatures
\sphinxAtStartPar
Registra un nuevo usuario en la base de datos.
\begin{quote}\begin{description}
\sphinxlineitem{Parámetros}\begin{itemize}
\item {} 
\sphinxAtStartPar
\sphinxstyleliteralstrong{\sphinxupquote{datos}} ({\hyperref[\detokenize{modelos:main.UsuarioRegistro}]{\sphinxcrossref{\sphinxstyleliteralemphasis{\sphinxupquote{UsuarioRegistro}}}}}) \textendash{} Objeto que contiene el nombre de usuario y la contraseña.

\item {} 
\sphinxAtStartPar
\sphinxstyleliteralstrong{\sphinxupquote{db}} (\sphinxstyleliteralemphasis{\sphinxupquote{Session}}) \textendash{} Sesión activa de la base de datos, proporcionada por FastAPI.

\end{itemize}

\sphinxlineitem{Devuelve}
\sphinxAtStartPar
Un mensaje indicando si el usuario fue registrado exitosamente.

\sphinxlineitem{Tipo del valor devuelto}
\sphinxAtStartPar
dict

\sphinxlineitem{Muestra}
\sphinxAtStartPar
\sphinxstyleliteralstrong{\sphinxupquote{HTTPException 400}} \textendash{} Si los campos son inválidos o el nombre de usuario ya existe.

\end{description}\end{quote}

\end{fulllineitems}

\index{saludo() (en el módulo main)@\spxentry{saludo()}\spxextra{en el módulo main}}

\begin{fulllineitems}
\phantomsection\label{\detokenize{endpoints:main.saludo}}
\pysigstartsignatures
\pysiglinewithargsret
{\sphinxbfcode{\sphinxupquote{\DUrole{k}{async}\DUrole{w}{ }}}\sphinxcode{\sphinxupquote{main.}}\sphinxbfcode{\sphinxupquote{saludo}}}
{}
{}
\pysigstopsignatures
\sphinxAtStartPar
Devuelve un mensaje simple para verificar que la API está activa.

\sphinxAtStartPar
Este endpoint puede utilizarse para pruebas de conectividad o para confirmar que el backend está desplegado correctamente.
\begin{quote}\begin{description}
\sphinxlineitem{Devuelve}
\sphinxAtStartPar
Mensaje de saludo indicando que la API funciona.

\sphinxlineitem{Tipo del valor devuelto}
\sphinxAtStartPar
dict

\end{description}\end{quote}

\end{fulllineitems}

\index{ver\_informe() (en el módulo main)@\spxentry{ver\_informe()}\spxextra{en el módulo main}}

\begin{fulllineitems}
\phantomsection\label{\detokenize{endpoints:main.ver_informe}}
\pysigstartsignatures
\pysiglinewithargsret
{\sphinxcode{\sphinxupquote{main.}}\sphinxbfcode{\sphinxupquote{ver\_informe}}}
{\sphinxparam{\DUrole{n}{token}\DUrole{p}{:}\DUrole{w}{ }\DUrole{n}{str}}\sphinxparamcomma \sphinxparam{\DUrole{n}{db}\DUrole{p}{:}\DUrole{w}{ }\DUrole{n}{Session}\DUrole{w}{ }\DUrole{o}{=}\DUrole{w}{ }\DUrole{default_value}{Depends(get\_db)}}}
{}
\pysigstopsignatures
\sphinxAtStartPar
Devuelve los datos del informe generado a partir de un token único.

\sphinxAtStartPar
Este endpoint permite el acceso público a un informe de diagnóstico de vehículo mediante un enlace con token generado previamente. No requiere autenticación, pero valida que el token sea legítimo.
\begin{quote}\begin{description}
\sphinxlineitem{Parámetros}
\sphinxAtStartPar
\sphinxstyleliteralstrong{\sphinxupquote{token}} (\sphinxstyleliteralemphasis{\sphinxupquote{str}}) \textendash{} Token único del informe generado.

\sphinxlineitem{Devuelve}
\sphinxAtStartPar
Información del vehículo (marca, modelo, año, etc.) y lista de errores DTC.

\sphinxlineitem{Tipo del valor devuelto}
\sphinxAtStartPar
dict

\sphinxlineitem{Muestra}\begin{itemize}
\item {} 
\sphinxAtStartPar
\sphinxstyleliteralstrong{\sphinxupquote{HTTPException 400}} \textendash{} Si el token no es válido o demasiado corto.

\item {} 
\sphinxAtStartPar
\sphinxstyleliteralstrong{\sphinxupquote{HTTPException 404}} \textendash{} Si no se encuentra el informe, el vehículo o los errores asociados.

\item {} 
\sphinxAtStartPar
\sphinxstyleliteralstrong{\sphinxupquote{HTTPException 500}} \textendash{} Si ocurre un error inesperado al procesar la solicitud.

\end{itemize}

\end{description}\end{quote}

\end{fulllineitems}

\index{verificar\_password() (en el módulo main)@\spxentry{verificar\_password()}\spxextra{en el módulo main}}

\begin{fulllineitems}
\phantomsection\label{\detokenize{endpoints:main.verificar_password}}
\pysigstartsignatures
\pysiglinewithargsret
{\sphinxcode{\sphinxupquote{main.}}\sphinxbfcode{\sphinxupquote{verificar\_password}}}
{\sphinxparam{\DUrole{n}{plain\_password}}\sphinxparamcomma \sphinxparam{\DUrole{n}{hashed\_password}}}
{}
\pysigstopsignatures
\sphinxAtStartPar
Verifica si una contraseña en texto plano coincide con su hash almacenado.
\begin{quote}\begin{description}
\sphinxlineitem{Parámetros}\begin{itemize}
\item {} 
\sphinxAtStartPar
\sphinxstyleliteralstrong{\sphinxupquote{password\_plano}} (\sphinxstyleliteralemphasis{\sphinxupquote{str}}) \textendash{} Contraseña proporcionada por el usuario.

\item {} 
\sphinxAtStartPar
\sphinxstyleliteralstrong{\sphinxupquote{password\_hash}} (\sphinxstyleliteralemphasis{\sphinxupquote{str}}) \textendash{} Hash almacenado en la base de datos.

\end{itemize}

\sphinxlineitem{Devuelve}
\sphinxAtStartPar
True si coinciden, False si no.

\sphinxlineitem{Tipo del valor devuelto}
\sphinxAtStartPar
bool

\end{description}\end{quote}

\end{fulllineitems}


\sphinxstepscope


\chapter{Modelos de Datos}
\label{\detokenize{modelos:modelos-de-datos}}\label{\detokenize{modelos::doc}}

\section{Modelos ORM (SQLAlchemy)}
\label{\detokenize{modelos:modelos-orm-sqlalchemy}}
\sphinxAtStartPar
Los modelos ORM representan las tablas en la base de datos. Están definidos con SQLAlchemy e incluyen relaciones entre entidades.
\begin{itemize}
\item {} 
\sphinxAtStartPar
\sphinxtitleref{Usuario}: representa un usuario registrado.

\item {} 
\sphinxAtStartPar
\sphinxtitleref{Vehiculo}: vehículo asociado a un usuario.

\item {} 
\sphinxAtStartPar
\sphinxtitleref{ErrorVehiculo}: errores OBD\sphinxhyphen{}II asociados a un vehículo.

\item {} 
\sphinxAtStartPar
\sphinxtitleref{InformeCompartido}: informes generados para ser enviados por correo.

\end{itemize}


\section{Modelos de validación (Pydantic)}
\label{\detokenize{modelos:modelos-de-validacion-pydantic}}
\sphinxAtStartPar
Los modelos Pydantic se utilizan para validar entradas y salidas en la API:
\begin{itemize}
\item {} 
\sphinxAtStartPar
\sphinxtitleref{UsuarioRegistro}, \sphinxtitleref{UsuarioLogin}: manejo de usuarios.

\item {} 
\sphinxAtStartPar
\sphinxtitleref{VehiculoRegistro}, \sphinxtitleref{VehiculoEdicion}: datos de los vehículos.

\item {} 
\sphinxAtStartPar
\sphinxtitleref{ErrorVehiculoRegistro}: errores recibidos desde el cliente OBD.

\item {} 
\sphinxAtStartPar
\sphinxtitleref{InformeRequest}: datos para enviar el informe por email.

\end{itemize}
\index{module@\spxentry{module}!main@\spxentry{main}}\index{main@\spxentry{main}!module@\spxentry{module}}\index{Base (clase en main)@\spxentry{Base}\spxextra{clase en main}}\phantomsection\label{\detokenize{modelos:module-main}}

\begin{fulllineitems}
\phantomsection\label{\detokenize{modelos:main.Base}}
\pysigstartsignatures
\pysiglinewithargsret
{\sphinxbfcode{\sphinxupquote{\DUrole{k}{class}\DUrole{w}{ }}}\sphinxcode{\sphinxupquote{main.}}\sphinxbfcode{\sphinxupquote{Base}}}
{\sphinxparam{\DUrole{o}{**}\DUrole{n}{kwargs}\DUrole{p}{:}\DUrole{w}{ }\DUrole{n}{Any}}}
{}
\pysigstopsignatures
\sphinxAtStartPar
Bases: \sphinxcode{\sphinxupquote{object}}

\sphinxAtStartPar
Configuración del sistema de envío de correos (FastAPI Mail):
\begin{itemize}
\item {} 
\sphinxAtStartPar
Las credenciales y parámetros se cargan desde variables de entorno.

\item {} 
\sphinxAtStartPar
\sphinxtitleref{FastMail} se instancia con esta configuración para ser usado en envíos.

\end{itemize}

\end{fulllineitems}

\index{ErrorVehiculo (clase en main)@\spxentry{ErrorVehiculo}\spxextra{clase en main}}

\begin{fulllineitems}
\phantomsection\label{\detokenize{modelos:main.ErrorVehiculo}}
\pysigstartsignatures
\pysiglinewithargsret
{\sphinxbfcode{\sphinxupquote{\DUrole{k}{class}\DUrole{w}{ }}}\sphinxcode{\sphinxupquote{main.}}\sphinxbfcode{\sphinxupquote{ErrorVehiculo}}}
{\sphinxparam{\DUrole{o}{**}\DUrole{n}{kwargs}}}
{}
\pysigstopsignatures
\sphinxAtStartPar
Bases: {\hyperref[\detokenize{modelos:main.Base}]{\sphinxcrossref{\sphinxcode{\sphinxupquote{Base}}}}}

\sphinxAtStartPar
Modelo ORM que almacena los errores OBD\sphinxhyphen{}II (códigos DTC) de un vehículo.
\begin{description}
\sphinxlineitem{Atributos:}
\sphinxAtStartPar
id (int): ID del error.
vehiculo\_id (int): ID del vehículo asociado.
codigo\_dtc (str): Código de diagnóstico (ej. P0301).

\sphinxlineitem{Relaciones:}
\sphinxAtStartPar
vehiculo (Vehiculo): Vehículo asociado.

\end{description}

\end{fulllineitems}

\index{ErrorVehiculoRegistro (clase en main)@\spxentry{ErrorVehiculoRegistro}\spxextra{clase en main}}

\begin{fulllineitems}
\phantomsection\label{\detokenize{modelos:main.ErrorVehiculoRegistro}}
\pysigstartsignatures
\pysiglinewithargsret
{\sphinxbfcode{\sphinxupquote{\DUrole{k}{class}\DUrole{w}{ }}}\sphinxcode{\sphinxupquote{main.}}\sphinxbfcode{\sphinxupquote{ErrorVehiculoRegistro}}}
{\sphinxparam{\DUrole{keyword-only-separator}{\DUrole{o}{\sphinxstyleabbreviation{*}}}}\sphinxparamcomma \sphinxparam{\DUrole{n}{codigo\_dtc}\DUrole{p}{:}\DUrole{w}{ }\DUrole{n}{list\DUrole{p}{{[}}str\DUrole{p}{{]}}}}\sphinxparamcomma \sphinxparam{\DUrole{n}{vehiculo\_id}\DUrole{p}{:}\DUrole{w}{ }\DUrole{n}{int}}}
{}
\pysigstopsignatures
\sphinxAtStartPar
Bases: \sphinxcode{\sphinxupquote{BaseModel}}

\sphinxAtStartPar
Modelo de solicitud para registrar errores OBD\sphinxhyphen{}II de un vehículo.
\begin{description}
\sphinxlineitem{Atributos:}
\sphinxAtStartPar
codigo\_dtc (list{[}str{]}): Lista de códigos DTC (códigos de diagnóstico).
vehiculo\_id (int): ID del vehículo al que se le asocian los errores.

\end{description}
\index{model\_config (atributo de main.ErrorVehiculoRegistro)@\spxentry{model\_config}\spxextra{atributo de main.ErrorVehiculoRegistro}}

\begin{fulllineitems}
\phantomsection\label{\detokenize{modelos:main.ErrorVehiculoRegistro.model_config}}
\pysigstartsignatures
\pysigline
{\sphinxbfcode{\sphinxupquote{model\_config}}\sphinxbfcode{\sphinxupquote{\DUrole{p}{:}\DUrole{w}{ }ClassVar\DUrole{p}{{[}}ConfigDict\DUrole{p}{{]}}}}\sphinxbfcode{\sphinxupquote{\DUrole{w}{ }\DUrole{p}{=}\DUrole{w}{ }\{\}}}}
\pysigstopsignatures
\sphinxAtStartPar
Configuration for the model, should be a dictionary conforming to {[}\sphinxtitleref{ConfigDict}{]}{[}pydantic.config.ConfigDict{]}.

\end{fulllineitems}


\end{fulllineitems}

\index{InformeCompartido (clase en main)@\spxentry{InformeCompartido}\spxextra{clase en main}}

\begin{fulllineitems}
\phantomsection\label{\detokenize{modelos:main.InformeCompartido}}
\pysigstartsignatures
\pysiglinewithargsret
{\sphinxbfcode{\sphinxupquote{\DUrole{k}{class}\DUrole{w}{ }}}\sphinxcode{\sphinxupquote{main.}}\sphinxbfcode{\sphinxupquote{InformeCompartido}}}
{\sphinxparam{\DUrole{o}{**}\DUrole{n}{kwargs}}}
{}
\pysigstopsignatures
\sphinxAtStartPar
Bases: {\hyperref[\detokenize{modelos:main.Base}]{\sphinxcrossref{\sphinxcode{\sphinxupquote{Base}}}}}

\sphinxAtStartPar
Modelo ORM que representa un informe compartido con un cliente por email.
\begin{description}
\sphinxlineitem{Atributos:}
\sphinxAtStartPar
id (int): ID del informe.
token (str): Token único para acceder al informe.
vehiculo\_id (int): ID del vehículo relacionado.
email\_cliente (str): Email al que se envía el informe.
creado\_en (str): Fecha y hora de creación del informe (ISO format).

\sphinxlineitem{Relaciones:}
\sphinxAtStartPar
vehiculo (Vehiculo): Vehículo asociado.

\end{description}

\end{fulllineitems}

\index{InformeRequest (clase en main)@\spxentry{InformeRequest}\spxextra{clase en main}}

\begin{fulllineitems}
\phantomsection\label{\detokenize{modelos:main.InformeRequest}}
\pysigstartsignatures
\pysiglinewithargsret
{\sphinxbfcode{\sphinxupquote{\DUrole{k}{class}\DUrole{w}{ }}}\sphinxcode{\sphinxupquote{main.}}\sphinxbfcode{\sphinxupquote{InformeRequest}}}
{\sphinxparam{\DUrole{keyword-only-separator}{\DUrole{o}{\sphinxstyleabbreviation{*}}}}\sphinxparamcomma \sphinxparam{\DUrole{n}{email}\DUrole{p}{:}\DUrole{w}{ }\DUrole{n}{str}}}
{}
\pysigstopsignatures
\sphinxAtStartPar
Bases: \sphinxcode{\sphinxupquote{BaseModel}}

\sphinxAtStartPar
Modelo de solicitud para generar y enviar un informe por correo.
\begin{description}
\sphinxlineitem{Atributos:}
\sphinxAtStartPar
email (str): Dirección de email del cliente destinatario.

\end{description}
\index{model\_config (atributo de main.InformeRequest)@\spxentry{model\_config}\spxextra{atributo de main.InformeRequest}}

\begin{fulllineitems}
\phantomsection\label{\detokenize{modelos:main.InformeRequest.model_config}}
\pysigstartsignatures
\pysigline
{\sphinxbfcode{\sphinxupquote{model\_config}}\sphinxbfcode{\sphinxupquote{\DUrole{p}{:}\DUrole{w}{ }ClassVar\DUrole{p}{{[}}ConfigDict\DUrole{p}{{]}}}}\sphinxbfcode{\sphinxupquote{\DUrole{w}{ }\DUrole{p}{=}\DUrole{w}{ }\{\}}}}
\pysigstopsignatures
\sphinxAtStartPar
Configuration for the model, should be a dictionary conforming to {[}\sphinxtitleref{ConfigDict}{]}{[}pydantic.config.ConfigDict{]}.

\end{fulllineitems}


\end{fulllineitems}

\index{Usuario (clase en main)@\spxentry{Usuario}\spxextra{clase en main}}

\begin{fulllineitems}
\phantomsection\label{\detokenize{modelos:main.Usuario}}
\pysigstartsignatures
\pysiglinewithargsret
{\sphinxbfcode{\sphinxupquote{\DUrole{k}{class}\DUrole{w}{ }}}\sphinxcode{\sphinxupquote{main.}}\sphinxbfcode{\sphinxupquote{Usuario}}}
{\sphinxparam{\DUrole{o}{**}\DUrole{n}{kwargs}}}
{}
\pysigstopsignatures
\sphinxAtStartPar
Bases: {\hyperref[\detokenize{modelos:main.Base}]{\sphinxcrossref{\sphinxcode{\sphinxupquote{Base}}}}}

\sphinxAtStartPar
Modelo ORM que representa a los usuarios del sistema.
\begin{description}
\sphinxlineitem{Atributos:}
\sphinxAtStartPar
id (int): ID autoincremental (clave primaria).
username (str): Nombre de usuario, único.
password\_hash (str): Contraseña hasheada con bcrypt.

\sphinxlineitem{Relaciones:}
\sphinxAtStartPar
vehiculos (List{[}Vehiculo{]}): Lista de vehículos registrados por el usuario.

\end{description}

\end{fulllineitems}

\index{UsuarioLogin (clase en main)@\spxentry{UsuarioLogin}\spxextra{clase en main}}

\begin{fulllineitems}
\phantomsection\label{\detokenize{modelos:main.UsuarioLogin}}
\pysigstartsignatures
\pysiglinewithargsret
{\sphinxbfcode{\sphinxupquote{\DUrole{k}{class}\DUrole{w}{ }}}\sphinxcode{\sphinxupquote{main.}}\sphinxbfcode{\sphinxupquote{UsuarioLogin}}}
{\sphinxparam{\DUrole{keyword-only-separator}{\DUrole{o}{\sphinxstyleabbreviation{*}}}}\sphinxparamcomma \sphinxparam{\DUrole{n}{username}\DUrole{p}{:}\DUrole{w}{ }\DUrole{n}{str}}\sphinxparamcomma \sphinxparam{\DUrole{n}{password}\DUrole{p}{:}\DUrole{w}{ }\DUrole{n}{str}}}
{}
\pysigstopsignatures
\sphinxAtStartPar
Bases: \sphinxcode{\sphinxupquote{BaseModel}}

\sphinxAtStartPar
Modelo de solicitud para iniciar sesión de usuario.
\begin{description}
\sphinxlineitem{Atributos:}
\sphinxAtStartPar
username (str): Nombre de usuario.
password (str): Contraseña en texto plano.

\end{description}
\index{model\_config (atributo de main.UsuarioLogin)@\spxentry{model\_config}\spxextra{atributo de main.UsuarioLogin}}

\begin{fulllineitems}
\phantomsection\label{\detokenize{modelos:main.UsuarioLogin.model_config}}
\pysigstartsignatures
\pysigline
{\sphinxbfcode{\sphinxupquote{model\_config}}\sphinxbfcode{\sphinxupquote{\DUrole{p}{:}\DUrole{w}{ }ClassVar\DUrole{p}{{[}}ConfigDict\DUrole{p}{{]}}}}\sphinxbfcode{\sphinxupquote{\DUrole{w}{ }\DUrole{p}{=}\DUrole{w}{ }\{\}}}}
\pysigstopsignatures
\sphinxAtStartPar
Configuration for the model, should be a dictionary conforming to {[}\sphinxtitleref{ConfigDict}{]}{[}pydantic.config.ConfigDict{]}.

\end{fulllineitems}


\end{fulllineitems}

\index{UsuarioRegistro (clase en main)@\spxentry{UsuarioRegistro}\spxextra{clase en main}}

\begin{fulllineitems}
\phantomsection\label{\detokenize{modelos:main.UsuarioRegistro}}
\pysigstartsignatures
\pysiglinewithargsret
{\sphinxbfcode{\sphinxupquote{\DUrole{k}{class}\DUrole{w}{ }}}\sphinxcode{\sphinxupquote{main.}}\sphinxbfcode{\sphinxupquote{UsuarioRegistro}}}
{\sphinxparam{\DUrole{keyword-only-separator}{\DUrole{o}{\sphinxstyleabbreviation{*}}}}\sphinxparamcomma \sphinxparam{\DUrole{n}{username}\DUrole{p}{:}\DUrole{w}{ }\DUrole{n}{str}}\sphinxparamcomma \sphinxparam{\DUrole{n}{password}\DUrole{p}{:}\DUrole{w}{ }\DUrole{n}{str}}}
{}
\pysigstopsignatures
\sphinxAtStartPar
Bases: \sphinxcode{\sphinxupquote{BaseModel}}

\sphinxAtStartPar
Modelo de solicitud para registrar un nuevo usuario.
\begin{description}
\sphinxlineitem{Atributos:}
\sphinxAtStartPar
username (str): Nombre de usuario.
password (str): Contraseña en texto plano.

\end{description}
\index{model\_config (atributo de main.UsuarioRegistro)@\spxentry{model\_config}\spxextra{atributo de main.UsuarioRegistro}}

\begin{fulllineitems}
\phantomsection\label{\detokenize{modelos:main.UsuarioRegistro.model_config}}
\pysigstartsignatures
\pysigline
{\sphinxbfcode{\sphinxupquote{model\_config}}\sphinxbfcode{\sphinxupquote{\DUrole{p}{:}\DUrole{w}{ }ClassVar\DUrole{p}{{[}}ConfigDict\DUrole{p}{{]}}}}\sphinxbfcode{\sphinxupquote{\DUrole{w}{ }\DUrole{p}{=}\DUrole{w}{ }\{\}}}}
\pysigstopsignatures
\sphinxAtStartPar
Configuration for the model, should be a dictionary conforming to {[}\sphinxtitleref{ConfigDict}{]}{[}pydantic.config.ConfigDict{]}.

\end{fulllineitems}


\end{fulllineitems}

\index{Vehiculo (clase en main)@\spxentry{Vehiculo}\spxextra{clase en main}}

\begin{fulllineitems}
\phantomsection\label{\detokenize{modelos:main.Vehiculo}}
\pysigstartsignatures
\pysiglinewithargsret
{\sphinxbfcode{\sphinxupquote{\DUrole{k}{class}\DUrole{w}{ }}}\sphinxcode{\sphinxupquote{main.}}\sphinxbfcode{\sphinxupquote{Vehiculo}}}
{\sphinxparam{\DUrole{o}{**}\DUrole{n}{kwargs}}}
{}
\pysigstopsignatures
\sphinxAtStartPar
Bases: {\hyperref[\detokenize{modelos:main.Base}]{\sphinxcrossref{\sphinxcode{\sphinxupquote{Base}}}}}

\sphinxAtStartPar
Modelo ORM que representa un vehículo registrado.
\begin{description}
\sphinxlineitem{Atributos:}
\sphinxAtStartPar
id (int): ID del vehículo.
marca (str): Marca del vehículo.
modelo (str): Modelo del vehículo.
year (int): Año de fabricación.
rpm (int): Revoluciones por minuto.
velocidad (int): Velocidad actual.
vin (str): Número VIN único del vehículo.
revision (str): Información de revisión técnica.
usuario\_id (int): ID del usuario al que pertenece el vehículo.

\sphinxlineitem{Relaciones:}
\sphinxAtStartPar
usuario (Usuario): Usuario propietario.
errores (List{[}ErrorVehiculo{]}): Lista de errores asociados.
informes\_compartidos (List{[}InformeCompartido{]}): Informes generados con token público.

\end{description}

\end{fulllineitems}

\index{VehiculoEdicion (clase en main)@\spxentry{VehiculoEdicion}\spxextra{clase en main}}

\begin{fulllineitems}
\phantomsection\label{\detokenize{modelos:main.VehiculoEdicion}}
\pysigstartsignatures
\pysiglinewithargsret
{\sphinxbfcode{\sphinxupquote{\DUrole{k}{class}\DUrole{w}{ }}}\sphinxcode{\sphinxupquote{main.}}\sphinxbfcode{\sphinxupquote{VehiculoEdicion}}}
{\sphinxparam{\DUrole{keyword-only-separator}{\DUrole{o}{\sphinxstyleabbreviation{*}}}}\sphinxparamcomma \sphinxparam{\DUrole{n}{marca}\DUrole{p}{:}\DUrole{w}{ }\DUrole{n}{str}}\sphinxparamcomma \sphinxparam{\DUrole{n}{modelo}\DUrole{p}{:}\DUrole{w}{ }\DUrole{n}{str}}\sphinxparamcomma \sphinxparam{\DUrole{n}{year}\DUrole{p}{:}\DUrole{w}{ }\DUrole{n}{int}}\sphinxparamcomma \sphinxparam{\DUrole{n}{rpm}\DUrole{p}{:}\DUrole{w}{ }\DUrole{n}{int}}\sphinxparamcomma \sphinxparam{\DUrole{n}{velocidad}\DUrole{p}{:}\DUrole{w}{ }\DUrole{n}{int}}\sphinxparamcomma \sphinxparam{\DUrole{n}{vin}\DUrole{p}{:}\DUrole{w}{ }\DUrole{n}{str}}}
{}
\pysigstopsignatures
\sphinxAtStartPar
Bases: \sphinxcode{\sphinxupquote{BaseModel}}

\sphinxAtStartPar
Modelo de solicitud para editar un vehículo existente.
\begin{description}
\sphinxlineitem{Atributos:}
\sphinxAtStartPar
marca (str): Marca del vehículo.
modelo (str): Modelo del vehículo.
year (int): Año de fabricación.
rpm (int): Revoluciones por minuto.
velocidad (int): Velocidad actual.
vin (str): Número VIN del vehículo.

\end{description}
\index{model\_config (atributo de main.VehiculoEdicion)@\spxentry{model\_config}\spxextra{atributo de main.VehiculoEdicion}}

\begin{fulllineitems}
\phantomsection\label{\detokenize{modelos:main.VehiculoEdicion.model_config}}
\pysigstartsignatures
\pysigline
{\sphinxbfcode{\sphinxupquote{model\_config}}\sphinxbfcode{\sphinxupquote{\DUrole{p}{:}\DUrole{w}{ }ClassVar\DUrole{p}{{[}}ConfigDict\DUrole{p}{{]}}}}\sphinxbfcode{\sphinxupquote{\DUrole{w}{ }\DUrole{p}{=}\DUrole{w}{ }\{\}}}}
\pysigstopsignatures
\sphinxAtStartPar
Configuration for the model, should be a dictionary conforming to {[}\sphinxtitleref{ConfigDict}{]}{[}pydantic.config.ConfigDict{]}.

\end{fulllineitems}


\end{fulllineitems}

\index{VehiculoRegistro (clase en main)@\spxentry{VehiculoRegistro}\spxextra{clase en main}}

\begin{fulllineitems}
\phantomsection\label{\detokenize{modelos:main.VehiculoRegistro}}
\pysigstartsignatures
\pysiglinewithargsret
{\sphinxbfcode{\sphinxupquote{\DUrole{k}{class}\DUrole{w}{ }}}\sphinxcode{\sphinxupquote{main.}}\sphinxbfcode{\sphinxupquote{VehiculoRegistro}}}
{\sphinxparam{\DUrole{keyword-only-separator}{\DUrole{o}{\sphinxstyleabbreviation{*}}}}\sphinxparamcomma \sphinxparam{\DUrole{n}{marca}\DUrole{p}{:}\DUrole{w}{ }\DUrole{n}{str}}\sphinxparamcomma \sphinxparam{\DUrole{n}{modelo}\DUrole{p}{:}\DUrole{w}{ }\DUrole{n}{str}}\sphinxparamcomma \sphinxparam{\DUrole{n}{year}\DUrole{p}{:}\DUrole{w}{ }\DUrole{n}{int}}\sphinxparamcomma \sphinxparam{\DUrole{n}{rpm}\DUrole{p}{:}\DUrole{w}{ }\DUrole{n}{int}}\sphinxparamcomma \sphinxparam{\DUrole{n}{velocidad}\DUrole{p}{:}\DUrole{w}{ }\DUrole{n}{int}}\sphinxparamcomma \sphinxparam{\DUrole{n}{vin}\DUrole{p}{:}\DUrole{w}{ }\DUrole{n}{str}}\sphinxparamcomma \sphinxparam{\DUrole{n}{revision}\DUrole{p}{:}\DUrole{w}{ }\DUrole{n}{dict}}}
{}
\pysigstopsignatures
\sphinxAtStartPar
Bases: \sphinxcode{\sphinxupquote{BaseModel}}

\sphinxAtStartPar
Modelo de solicitud para registrar un nuevo vehículo.
\begin{description}
\sphinxlineitem{Atributos:}
\sphinxAtStartPar
marca (str): Marca del vehículo.
modelo (str): Modelo del vehículo.
year (int): Año del vehículo.
rpm (int): RPM del motor.
velocidad (int): Velocidad del vehículo.
vin (str): Número VIN único del vehículo.
revision (dict): Detalles de la revisión técnica (estructura flexible).

\end{description}
\index{model\_config (atributo de main.VehiculoRegistro)@\spxentry{model\_config}\spxextra{atributo de main.VehiculoRegistro}}

\begin{fulllineitems}
\phantomsection\label{\detokenize{modelos:main.VehiculoRegistro.model_config}}
\pysigstartsignatures
\pysigline
{\sphinxbfcode{\sphinxupquote{model\_config}}\sphinxbfcode{\sphinxupquote{\DUrole{p}{:}\DUrole{w}{ }ClassVar\DUrole{p}{{[}}ConfigDict\DUrole{p}{{]}}}}\sphinxbfcode{\sphinxupquote{\DUrole{w}{ }\DUrole{p}{=}\DUrole{w}{ }\{\}}}}
\pysigstopsignatures
\sphinxAtStartPar
Configuration for the model, should be a dictionary conforming to {[}\sphinxtitleref{ConfigDict}{]}{[}pydantic.config.ConfigDict{]}.

\end{fulllineitems}


\end{fulllineitems}

\index{crear\_informe() (en el módulo main)@\spxentry{crear\_informe()}\spxextra{en el módulo main}}

\begin{fulllineitems}
\phantomsection\label{\detokenize{modelos:main.crear_informe}}
\pysigstartsignatures
\pysiglinewithargsret
{\sphinxbfcode{\sphinxupquote{\DUrole{k}{async}\DUrole{w}{ }}}\sphinxcode{\sphinxupquote{main.}}\sphinxbfcode{\sphinxupquote{crear\_informe}}}
{\sphinxparam{\DUrole{n}{vehiculo\_id}\DUrole{p}{:}\DUrole{w}{ }\DUrole{n}{int}}\sphinxparamcomma \sphinxparam{\DUrole{n}{request}\DUrole{p}{:}\DUrole{w}{ }\DUrole{n}{{\hyperref[\detokenize{modelos:main.InformeRequest}]{\sphinxcrossref{InformeRequest}}}}}\sphinxparamcomma \sphinxparam{\DUrole{n}{usuario}\DUrole{p}{:}\DUrole{w}{ }\DUrole{n}{{\hyperref[\detokenize{modelos:main.Usuario}]{\sphinxcrossref{Usuario}}}}\DUrole{w}{ }\DUrole{o}{=}\DUrole{w}{ }\DUrole{default_value}{Depends(obtener\_usuario\_desde\_token)}}\sphinxparamcomma \sphinxparam{\DUrole{n}{db}\DUrole{p}{:}\DUrole{w}{ }\DUrole{n}{Session}\DUrole{w}{ }\DUrole{o}{=}\DUrole{w}{ }\DUrole{default_value}{Depends(get\_db)}}}
{}
\pysigstopsignatures
\sphinxAtStartPar
Crea un informe de errores del vehículo y lo envía al email del cliente.

\sphinxAtStartPar
Este endpoint genera un enlace único que da acceso a una vista del informe de diagnóstico del vehículo. Se envía un correo al cliente con dicho enlace.
\begin{quote}\begin{description}
\sphinxlineitem{Parámetros}\begin{itemize}
\item {} 
\sphinxAtStartPar
\sphinxstyleliteralstrong{\sphinxupquote{vehiculo\_id}} (\sphinxstyleliteralemphasis{\sphinxupquote{int}}) \textendash{} ID del vehículo del que se desea generar el informe.

\item {} 
\sphinxAtStartPar
\sphinxstyleliteralstrong{\sphinxupquote{request}} ({\hyperref[\detokenize{modelos:main.InformeRequest}]{\sphinxcrossref{\sphinxstyleliteralemphasis{\sphinxupquote{InformeRequest}}}}}) \textendash{} Objeto que contiene el email del cliente.

\item {} 
\sphinxAtStartPar
\sphinxstyleliteralstrong{\sphinxupquote{usuario}} ({\hyperref[\detokenize{modelos:main.Usuario}]{\sphinxcrossref{\sphinxstyleliteralemphasis{\sphinxupquote{Usuario}}}}}) \textendash{} Usuario autenticado mediante JWT.

\item {} 
\sphinxAtStartPar
\sphinxstyleliteralstrong{\sphinxupquote{db}} (\sphinxstyleliteralemphasis{\sphinxupquote{Session}}) \textendash{} Sesión activa de la base de datos.

\end{itemize}

\sphinxlineitem{Devuelve}
\sphinxAtStartPar
Mensaje de éxito, token generado y enlace de acceso.

\sphinxlineitem{Tipo del valor devuelto}
\sphinxAtStartPar
dict

\sphinxlineitem{Muestra}\begin{itemize}
\item {} 
\sphinxAtStartPar
\sphinxstyleliteralstrong{\sphinxupquote{HTTPException 400}} \textendash{} Si el email no es válido.

\item {} 
\sphinxAtStartPar
\sphinxstyleliteralstrong{\sphinxupquote{HTTPException 404}} \textendash{} Si el vehículo no pertenece al usuario.

\item {} 
\sphinxAtStartPar
\sphinxstyleliteralstrong{\sphinxupquote{HTTPException 500}} \textendash{} Si ocurre un error al guardar el informe o enviar el correo.

\end{itemize}

\end{description}\end{quote}

\end{fulllineitems}

\index{crear\_token() (en el módulo main)@\spxentry{crear\_token()}\spxextra{en el módulo main}}

\begin{fulllineitems}
\phantomsection\label{\detokenize{modelos:main.crear_token}}
\pysigstartsignatures
\pysiglinewithargsret
{\sphinxcode{\sphinxupquote{main.}}\sphinxbfcode{\sphinxupquote{crear\_token}}}
{\sphinxparam{\DUrole{n}{data}\DUrole{p}{:}\DUrole{w}{ }\DUrole{n}{dict}}\sphinxparamcomma \sphinxparam{\DUrole{n}{expira\_en}\DUrole{p}{:}\DUrole{w}{ }\DUrole{n}{int}\DUrole{w}{ }\DUrole{o}{=}\DUrole{w}{ }\DUrole{default_value}{300}}}
{}
\pysigstopsignatures
\sphinxAtStartPar
Genera un token JWT con los datos proporcionados y un tiempo de expiración opcional.
\begin{quote}\begin{description}
\sphinxlineitem{Parámetros}\begin{itemize}
\item {} 
\sphinxAtStartPar
\sphinxstyleliteralstrong{\sphinxupquote{datos}} (\sphinxstyleliteralemphasis{\sphinxupquote{dict}}) \textendash{} Datos a incluir en el payload del token.

\item {} 
\sphinxAtStartPar
\sphinxstyleliteralstrong{\sphinxupquote{tiempo\_expiracion}} (\sphinxstyleliteralemphasis{\sphinxupquote{Optional}}\sphinxstyleliteralemphasis{\sphinxupquote{{[}}}\sphinxstyleliteralemphasis{\sphinxupquote{timedelta}}\sphinxstyleliteralemphasis{\sphinxupquote{{]}}}) \textendash{} Tiempo personalizado de expiración. Si no se especifica, se usarán 30 minutos por defecto.

\end{itemize}

\sphinxlineitem{Devuelve}
\sphinxAtStartPar
Token JWT firmado.

\sphinxlineitem{Tipo del valor devuelto}
\sphinxAtStartPar
str

\sphinxlineitem{Muestra}
\sphinxAtStartPar
\sphinxstyleliteralstrong{\sphinxupquote{Exception}} \textendash{} Si hay un error al codificar el token.

\end{description}\end{quote}

\end{fulllineitems}

\index{editar\_vehiculo() (en el módulo main)@\spxentry{editar\_vehiculo()}\spxextra{en el módulo main}}

\begin{fulllineitems}
\phantomsection\label{\detokenize{modelos:main.editar_vehiculo}}
\pysigstartsignatures
\pysiglinewithargsret
{\sphinxcode{\sphinxupquote{main.}}\sphinxbfcode{\sphinxupquote{editar\_vehiculo}}}
{\sphinxparam{\DUrole{n}{vehiculo\_id}\DUrole{p}{:}\DUrole{w}{ }\DUrole{n}{int}}\sphinxparamcomma \sphinxparam{\DUrole{n}{datos}\DUrole{p}{:}\DUrole{w}{ }\DUrole{n}{{\hyperref[\detokenize{modelos:main.VehiculoEdicion}]{\sphinxcrossref{VehiculoEdicion}}}}}\sphinxparamcomma \sphinxparam{\DUrole{n}{usuario}\DUrole{p}{:}\DUrole{w}{ }\DUrole{n}{{\hyperref[\detokenize{modelos:main.Usuario}]{\sphinxcrossref{Usuario}}}}\DUrole{w}{ }\DUrole{o}{=}\DUrole{w}{ }\DUrole{default_value}{Depends(obtener\_usuario\_desde\_token)}}\sphinxparamcomma \sphinxparam{\DUrole{n}{db}\DUrole{p}{:}\DUrole{w}{ }\DUrole{n}{Session}\DUrole{w}{ }\DUrole{o}{=}\DUrole{w}{ }\DUrole{default_value}{Depends(get\_db)}}}
{}
\pysigstopsignatures
\sphinxAtStartPar
Actualiza los datos de un vehículo existente del usuario autenticado.
\begin{quote}\begin{description}
\sphinxlineitem{Parámetros}\begin{itemize}
\item {} 
\sphinxAtStartPar
\sphinxstyleliteralstrong{\sphinxupquote{vehiculo\_id}} (\sphinxstyleliteralemphasis{\sphinxupquote{int}}) \textendash{} ID del vehículo a modificar.

\item {} 
\sphinxAtStartPar
\sphinxstyleliteralstrong{\sphinxupquote{datos\_actualizados}} (\sphinxstyleliteralemphasis{\sphinxupquote{VehiculoBase}}) \textendash{} Nuevos datos del vehículo.

\item {} 
\sphinxAtStartPar
\sphinxstyleliteralstrong{\sphinxupquote{db}} (\sphinxstyleliteralemphasis{\sphinxupquote{Session}}) \textendash{} Sesión de base de datos.

\item {} 
\sphinxAtStartPar
\sphinxstyleliteralstrong{\sphinxupquote{usuario}} ({\hyperref[\detokenize{modelos:main.Usuario}]{\sphinxcrossref{\sphinxstyleliteralemphasis{\sphinxupquote{Usuario}}}}}) \textendash{} Usuario autenticado.

\end{itemize}

\sphinxlineitem{Devuelve}
\sphinxAtStartPar
Mensaje de éxito.

\sphinxlineitem{Tipo del valor devuelto}
\sphinxAtStartPar
dict

\sphinxlineitem{Muestra}
\sphinxAtStartPar
\sphinxstyleliteralstrong{\sphinxupquote{HTTPException 404}} \textendash{} Si el vehículo no existe o no pertenece al usuario.

\end{description}\end{quote}

\end{fulllineitems}

\index{eliminar\_vehiculo() (en el módulo main)@\spxentry{eliminar\_vehiculo()}\spxextra{en el módulo main}}

\begin{fulllineitems}
\phantomsection\label{\detokenize{modelos:main.eliminar_vehiculo}}
\pysigstartsignatures
\pysiglinewithargsret
{\sphinxcode{\sphinxupquote{main.}}\sphinxbfcode{\sphinxupquote{eliminar\_vehiculo}}}
{\sphinxparam{\DUrole{n}{vehiculo\_id}\DUrole{p}{:}\DUrole{w}{ }\DUrole{n}{int}}\sphinxparamcomma \sphinxparam{\DUrole{n}{usuario}\DUrole{p}{:}\DUrole{w}{ }\DUrole{n}{{\hyperref[\detokenize{modelos:main.Usuario}]{\sphinxcrossref{Usuario}}}}\DUrole{w}{ }\DUrole{o}{=}\DUrole{w}{ }\DUrole{default_value}{Depends(obtener\_usuario\_desde\_token)}}\sphinxparamcomma \sphinxparam{\DUrole{n}{db}\DUrole{p}{:}\DUrole{w}{ }\DUrole{n}{Session}\DUrole{w}{ }\DUrole{o}{=}\DUrole{w}{ }\DUrole{default_value}{Depends(get\_db)}}}
{}
\pysigstopsignatures
\sphinxAtStartPar
Elimina un vehículo registrado por el usuario autenticado.
\begin{quote}\begin{description}
\sphinxlineitem{Parámetros}\begin{itemize}
\item {} 
\sphinxAtStartPar
\sphinxstyleliteralstrong{\sphinxupquote{vehiculo\_id}} (\sphinxstyleliteralemphasis{\sphinxupquote{int}}) \textendash{} ID del vehículo a eliminar.

\item {} 
\sphinxAtStartPar
\sphinxstyleliteralstrong{\sphinxupquote{db}} (\sphinxstyleliteralemphasis{\sphinxupquote{Session}}) \textendash{} Sesión de base de datos.

\item {} 
\sphinxAtStartPar
\sphinxstyleliteralstrong{\sphinxupquote{usuario}} ({\hyperref[\detokenize{modelos:main.Usuario}]{\sphinxcrossref{\sphinxstyleliteralemphasis{\sphinxupquote{Usuario}}}}}) \textendash{} Usuario autenticado mediante JWT.

\end{itemize}

\sphinxlineitem{Devuelve}
\sphinxAtStartPar
Mensaje de éxito.

\sphinxlineitem{Tipo del valor devuelto}
\sphinxAtStartPar
dict

\sphinxlineitem{Muestra}
\sphinxAtStartPar
\sphinxstyleliteralstrong{\sphinxupquote{HTTPException 404}} \textendash{} Si el vehículo no existe o no pertenece al usuario.

\end{description}\end{quote}

\end{fulllineitems}

\index{fm (en el módulo main)@\spxentry{fm}\spxextra{en el módulo main}}

\begin{fulllineitems}
\phantomsection\label{\detokenize{modelos:main.fm}}
\pysigstartsignatures
\pysigline
{\sphinxcode{\sphinxupquote{main.}}\sphinxbfcode{\sphinxupquote{fm}}\sphinxbfcode{\sphinxupquote{\DUrole{w}{ }\DUrole{p}{=}\DUrole{w}{ }\textless{}fastapi\_mail.fastmail.FastMail object\textgreater{}}}}
\pysigstopsignatures
\sphinxAtStartPar
Configuración de seguridad:
\begin{itemize}
\item {} 
\sphinxAtStartPar
\sphinxtitleref{SECRET\_KEY}, \sphinxtitleref{ALGORITHM} y tiempo de expiración definen la seguridad del JWT.

\item {} 
\sphinxAtStartPar
\sphinxtitleref{pwd\_context} se usa para hashear contraseñas con bcrypt.

\item {} 
\sphinxAtStartPar
\sphinxtitleref{oauth2\_scheme} se usa como dependencia para extraer el token del header Authorization.

\end{itemize}

\end{fulllineitems}

\index{get\_car\_image() (en el módulo main)@\spxentry{get\_car\_image()}\spxextra{en el módulo main}}

\begin{fulllineitems}
\phantomsection\label{\detokenize{modelos:main.get_car_image}}
\pysigstartsignatures
\pysiglinewithargsret
{\sphinxcode{\sphinxupquote{main.}}\sphinxbfcode{\sphinxupquote{get\_car\_image}}}
{\sphinxparam{\DUrole{n}{searchTerm}\DUrole{p}{:}\DUrole{w}{ }\DUrole{n}{str}}}
{}
\pysigstopsignatures
\sphinxAtStartPar
Obtiene una URL de imagen representativa de un vehículo usando el término de búsqueda proporcionado.

\sphinxAtStartPar
Este endpoint consulta la API externa de carimagery.com para devolver la URL de una imagen que coincida con el término (por ejemplo, «Toyota Corolla 2020»).
\begin{quote}\begin{description}
\sphinxlineitem{Parámetros}
\sphinxAtStartPar
\sphinxstyleliteralstrong{\sphinxupquote{searchTerm}} (\sphinxstyleliteralemphasis{\sphinxupquote{str}}) \textendash{} Término de búsqueda del vehículo (marca, modelo, año, etc.).

\sphinxlineitem{Devuelve}
\sphinxAtStartPar
URL de la imagen del vehículo.

\sphinxlineitem{Tipo del valor devuelto}
\sphinxAtStartPar
str

\sphinxlineitem{Muestra}
\sphinxAtStartPar
\sphinxstyleliteralstrong{\sphinxupquote{HTTPException 500}} \textendash{} Si hay un error al consultar la API externa.

\end{description}\end{quote}

\end{fulllineitems}

\index{guardar\_errores() (en el módulo main)@\spxentry{guardar\_errores()}\spxextra{en el módulo main}}

\begin{fulllineitems}
\phantomsection\label{\detokenize{modelos:main.guardar_errores}}
\pysigstartsignatures
\pysiglinewithargsret
{\sphinxcode{\sphinxupquote{main.}}\sphinxbfcode{\sphinxupquote{guardar\_errores}}}
{\sphinxparam{\DUrole{n}{datos}\DUrole{p}{:}\DUrole{w}{ }\DUrole{n}{{\hyperref[\detokenize{modelos:main.ErrorVehiculoRegistro}]{\sphinxcrossref{ErrorVehiculoRegistro}}}}}\sphinxparamcomma \sphinxparam{\DUrole{n}{usuario}\DUrole{p}{:}\DUrole{w}{ }\DUrole{n}{{\hyperref[\detokenize{modelos:main.Usuario}]{\sphinxcrossref{Usuario}}}}\DUrole{w}{ }\DUrole{o}{=}\DUrole{w}{ }\DUrole{default_value}{Depends(obtener\_usuario\_desde\_token)}}\sphinxparamcomma \sphinxparam{\DUrole{n}{db}\DUrole{p}{:}\DUrole{w}{ }\DUrole{n}{Session}\DUrole{w}{ }\DUrole{o}{=}\DUrole{w}{ }\DUrole{default_value}{Depends(get\_db)}}}
{}
\pysigstopsignatures
\sphinxAtStartPar
Guarda una lista de códigos de error OBD\sphinxhyphen{}II (DTC) asociados a un vehículo del usuario autenticado.

\sphinxAtStartPar
Este endpoint es utilizado por el cliente Python que recibe errores del escáner OBD\sphinxhyphen{}II y los envía al backend para su almacenamiento.
\begin{quote}\begin{description}
\sphinxlineitem{Parámetros}\begin{itemize}
\item {} 
\sphinxAtStartPar
\sphinxstyleliteralstrong{\sphinxupquote{datos}} ({\hyperref[\detokenize{modelos:main.ErrorVehiculoRegistro}]{\sphinxcrossref{\sphinxstyleliteralemphasis{\sphinxupquote{ErrorVehiculoRegistro}}}}}) \textendash{} Objeto que contiene el ID del vehículo y una lista de códigos DTC.

\item {} 
\sphinxAtStartPar
\sphinxstyleliteralstrong{\sphinxupquote{usuario}} ({\hyperref[\detokenize{modelos:main.Usuario}]{\sphinxcrossref{\sphinxstyleliteralemphasis{\sphinxupquote{Usuario}}}}}) \textendash{} Usuario autenticado, obtenido desde el token JWT.

\item {} 
\sphinxAtStartPar
\sphinxstyleliteralstrong{\sphinxupquote{db}} (\sphinxstyleliteralemphasis{\sphinxupquote{Session}}) \textendash{} Sesión activa de la base de datos.

\end{itemize}

\sphinxlineitem{Devuelve}
\sphinxAtStartPar
Mensaje de confirmación si los errores fueron guardados correctamente.

\sphinxlineitem{Tipo del valor devuelto}
\sphinxAtStartPar
dict

\sphinxlineitem{Muestra}\begin{itemize}
\item {} 
\sphinxAtStartPar
\sphinxstyleliteralstrong{\sphinxupquote{HTTPException 400}} \textendash{} \begin{itemize}
\item {} 
\sphinxAtStartPar
Si el ID del vehículo no es válido (no entero o negativo).
    \sphinxhyphen{} Si la lista de códigos está vacía o contiene valores vacíos.
    \sphinxhyphen{} Si hay códigos DTC duplicados.

\end{itemize}


\item {} 
\sphinxAtStartPar
\sphinxstyleliteralstrong{\sphinxupquote{HTTPException 404}} \textendash{} Si el vehículo no pertenece al usuario autenticado.

\item {} 
\sphinxAtStartPar
\sphinxstyleliteralstrong{\sphinxupquote{HTTPException 500}} \textendash{} Si ocurre un error inesperado al guardar en la base de datos.

\end{itemize}

\end{description}\end{quote}

\end{fulllineitems}

\index{guardar\_vehiculo() (en el módulo main)@\spxentry{guardar\_vehiculo()}\spxextra{en el módulo main}}

\begin{fulllineitems}
\phantomsection\label{\detokenize{modelos:main.guardar_vehiculo}}
\pysigstartsignatures
\pysiglinewithargsret
{\sphinxcode{\sphinxupquote{main.}}\sphinxbfcode{\sphinxupquote{guardar\_vehiculo}}}
{\sphinxparam{\DUrole{n}{datos}\DUrole{p}{:}\DUrole{w}{ }\DUrole{n}{{\hyperref[\detokenize{modelos:main.VehiculoRegistro}]{\sphinxcrossref{VehiculoRegistro}}}}}\sphinxparamcomma \sphinxparam{\DUrole{n}{usuario}\DUrole{p}{:}\DUrole{w}{ }\DUrole{n}{{\hyperref[\detokenize{modelos:main.Usuario}]{\sphinxcrossref{Usuario}}}}\DUrole{w}{ }\DUrole{o}{=}\DUrole{w}{ }\DUrole{default_value}{Depends(obtener\_usuario\_desde\_token)}}\sphinxparamcomma \sphinxparam{\DUrole{n}{db}\DUrole{p}{:}\DUrole{w}{ }\DUrole{n}{Session}\DUrole{w}{ }\DUrole{o}{=}\DUrole{w}{ }\DUrole{default_value}{Depends(get\_db)}}}
{}
\pysigstopsignatures
\sphinxAtStartPar
Guarda un nuevo vehículo en la base de datos asociado al usuario autenticado.
\begin{quote}\begin{description}
\sphinxlineitem{Parámetros}\begin{itemize}
\item {} 
\sphinxAtStartPar
\sphinxstyleliteralstrong{\sphinxupquote{vehiculo}} (\sphinxstyleliteralemphasis{\sphinxupquote{VehiculoBase}}) \textendash{} Datos del vehículo (marca, modelo, año, color, etc.).

\item {} 
\sphinxAtStartPar
\sphinxstyleliteralstrong{\sphinxupquote{db}} (\sphinxstyleliteralemphasis{\sphinxupquote{Session}}) \textendash{} Sesión de base de datos.

\item {} 
\sphinxAtStartPar
\sphinxstyleliteralstrong{\sphinxupquote{usuario}} ({\hyperref[\detokenize{modelos:main.Usuario}]{\sphinxcrossref{\sphinxstyleliteralemphasis{\sphinxupquote{Usuario}}}}}) \textendash{} Usuario autenticado, extraído desde el token JWT.

\end{itemize}

\sphinxlineitem{Devuelve}
\sphinxAtStartPar
Mensaje de confirmación.

\sphinxlineitem{Tipo del valor devuelto}
\sphinxAtStartPar
dict

\sphinxlineitem{Muestra}
\sphinxAtStartPar
\sphinxstyleliteralstrong{\sphinxupquote{HTTPException 401}} \textendash{} Si no se proporciona un token válido.

\end{description}\end{quote}

\end{fulllineitems}

\index{login() (en el módulo main)@\spxentry{login()}\spxextra{en el módulo main}}

\begin{fulllineitems}
\phantomsection\label{\detokenize{modelos:main.login}}
\pysigstartsignatures
\pysiglinewithargsret
{\sphinxcode{\sphinxupquote{main.}}\sphinxbfcode{\sphinxupquote{login}}}
{\sphinxparam{\DUrole{n}{datos}\DUrole{p}{:}\DUrole{w}{ }\DUrole{n}{{\hyperref[\detokenize{modelos:main.UsuarioLogin}]{\sphinxcrossref{UsuarioLogin}}}}}\sphinxparamcomma \sphinxparam{\DUrole{n}{db}\DUrole{p}{:}\DUrole{w}{ }\DUrole{n}{Session}\DUrole{w}{ }\DUrole{o}{=}\DUrole{w}{ }\DUrole{default_value}{Depends(get\_db)}}}
{}
\pysigstopsignatures
\sphinxAtStartPar
Autentica al usuario y devuelve un token JWT válido.
\begin{quote}\begin{description}
\sphinxlineitem{Parámetros}\begin{itemize}
\item {} 
\sphinxAtStartPar
\sphinxstyleliteralstrong{\sphinxupquote{datos}} ({\hyperref[\detokenize{modelos:main.UsuarioLogin}]{\sphinxcrossref{\sphinxstyleliteralemphasis{\sphinxupquote{UsuarioLogin}}}}}) \textendash{} Credenciales de usuario.

\item {} 
\sphinxAtStartPar
\sphinxstyleliteralstrong{\sphinxupquote{db}} (\sphinxstyleliteralemphasis{\sphinxupquote{Session}}) \textendash{} Sesión activa de la base de datos.

\end{itemize}

\sphinxlineitem{Devuelve}
\sphinxAtStartPar
Token JWT si la autenticación fue exitosa.

\sphinxlineitem{Tipo del valor devuelto}
\sphinxAtStartPar
dict

\sphinxlineitem{Muestra}\begin{itemize}
\item {} 
\sphinxAtStartPar
\sphinxstyleliteralstrong{\sphinxupquote{HTTPException 400}} \textendash{} Datos inválidos.

\item {} 
\sphinxAtStartPar
\sphinxstyleliteralstrong{\sphinxupquote{HTTPException 401}} \textendash{} Usuario no encontrado o contraseña incorrecta.

\item {} 
\sphinxAtStartPar
\sphinxstyleliteralstrong{\sphinxupquote{HTTPException 500}} \textendash{} Error al generar el token.

\end{itemize}

\end{description}\end{quote}

\end{fulllineitems}

\index{obtener\_errores() (en el módulo main)@\spxentry{obtener\_errores()}\spxextra{en el módulo main}}

\begin{fulllineitems}
\phantomsection\label{\detokenize{modelos:main.obtener_errores}}
\pysigstartsignatures
\pysiglinewithargsret
{\sphinxcode{\sphinxupquote{main.}}\sphinxbfcode{\sphinxupquote{obtener\_errores}}}
{\sphinxparam{\DUrole{n}{vehiculo\_id}\DUrole{p}{:}\DUrole{w}{ }\DUrole{n}{int}}\sphinxparamcomma \sphinxparam{\DUrole{n}{usuario}\DUrole{p}{:}\DUrole{w}{ }\DUrole{n}{{\hyperref[\detokenize{modelos:main.Usuario}]{\sphinxcrossref{Usuario}}}}\DUrole{w}{ }\DUrole{o}{=}\DUrole{w}{ }\DUrole{default_value}{Depends(obtener\_usuario\_desde\_token)}}\sphinxparamcomma \sphinxparam{\DUrole{n}{db}\DUrole{p}{:}\DUrole{w}{ }\DUrole{n}{Session}\DUrole{w}{ }\DUrole{o}{=}\DUrole{w}{ }\DUrole{default_value}{Depends(get\_db)}}}
{}
\pysigstopsignatures
\sphinxAtStartPar
Devuelve todos los errores DTC (códigos OBD\sphinxhyphen{}II) asociados a un vehículo del usuario autenticado.
\begin{quote}\begin{description}
\sphinxlineitem{Parámetros}\begin{itemize}
\item {} 
\sphinxAtStartPar
\sphinxstyleliteralstrong{\sphinxupquote{vehiculo\_id}} (\sphinxstyleliteralemphasis{\sphinxupquote{int}}) \textendash{} ID del vehículo para el que se desean consultar los errores.

\item {} 
\sphinxAtStartPar
\sphinxstyleliteralstrong{\sphinxupquote{usuario}} ({\hyperref[\detokenize{modelos:main.Usuario}]{\sphinxcrossref{\sphinxstyleliteralemphasis{\sphinxupquote{Usuario}}}}}) \textendash{} Usuario autenticado mediante JWT.

\item {} 
\sphinxAtStartPar
\sphinxstyleliteralstrong{\sphinxupquote{db}} (\sphinxstyleliteralemphasis{\sphinxupquote{Session}}) \textendash{} Sesión activa de la base de datos.

\end{itemize}

\sphinxlineitem{Devuelve}
\sphinxAtStartPar
Lista de errores registrados.

\sphinxlineitem{Tipo del valor devuelto}
\sphinxAtStartPar
List{[}{\hyperref[\detokenize{modelos:main.ErrorVehiculo}]{\sphinxcrossref{ErrorVehiculo}}}{]}

\sphinxlineitem{Muestra}
\sphinxAtStartPar
\sphinxstyleliteralstrong{\sphinxupquote{HTTPException 404}} \textendash{} Si no existen errores para ese vehículo.

\end{description}\end{quote}

\end{fulllineitems}

\index{obtener\_usuario\_desde\_token() (en el módulo main)@\spxentry{obtener\_usuario\_desde\_token()}\spxextra{en el módulo main}}

\begin{fulllineitems}
\phantomsection\label{\detokenize{modelos:main.obtener_usuario_desde_token}}
\pysigstartsignatures
\pysiglinewithargsret
{\sphinxcode{\sphinxupquote{main.}}\sphinxbfcode{\sphinxupquote{obtener\_usuario\_desde\_token}}}
{\sphinxparam{\DUrole{n}{token}\DUrole{p}{:}\DUrole{w}{ }\DUrole{n}{str}\DUrole{w}{ }\DUrole{o}{=}\DUrole{w}{ }\DUrole{default_value}{Depends(OAuth2PasswordBearer)}}\sphinxparamcomma \sphinxparam{\DUrole{n}{db}\DUrole{p}{:}\DUrole{w}{ }\DUrole{n}{Session}\DUrole{w}{ }\DUrole{o}{=}\DUrole{w}{ }\DUrole{default_value}{Depends(get\_db)}}}
{}
\pysigstopsignatures
\sphinxAtStartPar
Extrae y valida el usuario actual a partir del token JWT proporcionado.
\begin{quote}\begin{description}
\sphinxlineitem{Parámetros}\begin{itemize}
\item {} 
\sphinxAtStartPar
\sphinxstyleliteralstrong{\sphinxupquote{token}} (\sphinxstyleliteralemphasis{\sphinxupquote{str}}) \textendash{} Token JWT incluido en el encabezado de autorización.

\item {} 
\sphinxAtStartPar
\sphinxstyleliteralstrong{\sphinxupquote{db}} (\sphinxstyleliteralemphasis{\sphinxupquote{Session}}) \textendash{} Sesión de base de datos.

\end{itemize}

\sphinxlineitem{Devuelve}
\sphinxAtStartPar
Instancia del usuario autenticado.

\sphinxlineitem{Tipo del valor devuelto}
\sphinxAtStartPar
{\hyperref[\detokenize{modelos:main.Usuario}]{\sphinxcrossref{Usuario}}}

\sphinxlineitem{Muestra}
\sphinxAtStartPar
\sphinxstyleliteralstrong{\sphinxupquote{HTTPException 401}} \textendash{} Si el token es inválido o ha expirado.

\end{description}\end{quote}

\end{fulllineitems}

\index{obtener\_vehiculo() (en el módulo main)@\spxentry{obtener\_vehiculo()}\spxextra{en el módulo main}}

\begin{fulllineitems}
\phantomsection\label{\detokenize{modelos:main.obtener_vehiculo}}
\pysigstartsignatures
\pysiglinewithargsret
{\sphinxcode{\sphinxupquote{main.}}\sphinxbfcode{\sphinxupquote{obtener\_vehiculo}}}
{\sphinxparam{\DUrole{n}{vehiculo\_id}\DUrole{p}{:}\DUrole{w}{ }\DUrole{n}{int}}\sphinxparamcomma \sphinxparam{\DUrole{n}{usuario}\DUrole{p}{:}\DUrole{w}{ }\DUrole{n}{{\hyperref[\detokenize{modelos:main.Usuario}]{\sphinxcrossref{Usuario}}}}\DUrole{w}{ }\DUrole{o}{=}\DUrole{w}{ }\DUrole{default_value}{Depends(obtener\_usuario\_desde\_token)}}\sphinxparamcomma \sphinxparam{\DUrole{n}{db}\DUrole{p}{:}\DUrole{w}{ }\DUrole{n}{Session}\DUrole{w}{ }\DUrole{o}{=}\DUrole{w}{ }\DUrole{default_value}{Depends(get\_db)}}}
{}
\pysigstopsignatures
\sphinxAtStartPar
Recupera la información de un vehículo específico registrado por el usuario autenticado.
\begin{quote}\begin{description}
\sphinxlineitem{Parámetros}\begin{itemize}
\item {} 
\sphinxAtStartPar
\sphinxstyleliteralstrong{\sphinxupquote{vehiculo\_id}} (\sphinxstyleliteralemphasis{\sphinxupquote{int}}) \textendash{} ID del vehículo a consultar.

\item {} 
\sphinxAtStartPar
\sphinxstyleliteralstrong{\sphinxupquote{usuario}} ({\hyperref[\detokenize{modelos:main.Usuario}]{\sphinxcrossref{\sphinxstyleliteralemphasis{\sphinxupquote{Usuario}}}}}) \textendash{} Usuario autenticado mediante JWT.

\item {} 
\sphinxAtStartPar
\sphinxstyleliteralstrong{\sphinxupquote{db}} (\sphinxstyleliteralemphasis{\sphinxupquote{Session}}) \textendash{} Sesión activa de la base de datos.

\end{itemize}

\sphinxlineitem{Devuelve}
\sphinxAtStartPar
Objeto del vehículo solicitado.

\sphinxlineitem{Tipo del valor devuelto}
\sphinxAtStartPar
{\hyperref[\detokenize{modelos:main.Vehiculo}]{\sphinxcrossref{Vehiculo}}}

\sphinxlineitem{Muestra}
\sphinxAtStartPar
\sphinxstyleliteralstrong{\sphinxupquote{HTTPException 404}} \textendash{} Si el vehículo no pertenece al usuario o no existe.

\end{description}\end{quote}

\end{fulllineitems}

\index{obtener\_vehiculos() (en el módulo main)@\spxentry{obtener\_vehiculos()}\spxextra{en el módulo main}}

\begin{fulllineitems}
\phantomsection\label{\detokenize{modelos:main.obtener_vehiculos}}
\pysigstartsignatures
\pysiglinewithargsret
{\sphinxcode{\sphinxupquote{main.}}\sphinxbfcode{\sphinxupquote{obtener\_vehiculos}}}
{\sphinxparam{\DUrole{n}{usuario}\DUrole{p}{:}\DUrole{w}{ }\DUrole{n}{{\hyperref[\detokenize{modelos:main.Usuario}]{\sphinxcrossref{Usuario}}}}\DUrole{w}{ }\DUrole{o}{=}\DUrole{w}{ }\DUrole{default_value}{Depends(obtener\_usuario\_desde\_token)}}\sphinxparamcomma \sphinxparam{\DUrole{n}{db}\DUrole{p}{:}\DUrole{w}{ }\DUrole{n}{Session}\DUrole{w}{ }\DUrole{o}{=}\DUrole{w}{ }\DUrole{default_value}{Depends(get\_db)}}}
{}
\pysigstopsignatures
\sphinxAtStartPar
Obtiene todos los vehículos registrados por el usuario autenticado.
\begin{quote}\begin{description}
\sphinxlineitem{Parámetros}\begin{itemize}
\item {} 
\sphinxAtStartPar
\sphinxstyleliteralstrong{\sphinxupquote{db}} (\sphinxstyleliteralemphasis{\sphinxupquote{Session}}) \textendash{} Sesión de base de datos.

\item {} 
\sphinxAtStartPar
\sphinxstyleliteralstrong{\sphinxupquote{usuario}} ({\hyperref[\detokenize{modelos:main.Usuario}]{\sphinxcrossref{\sphinxstyleliteralemphasis{\sphinxupquote{Usuario}}}}}) \textendash{} Usuario autenticado mediante JWT.

\end{itemize}

\sphinxlineitem{Devuelve}
\sphinxAtStartPar
Lista de vehículos asociados al usuario.

\sphinxlineitem{Tipo del valor devuelto}
\sphinxAtStartPar
List{[}VehiculoBase{]}

\end{description}\end{quote}

\end{fulllineitems}

\index{register() (en el módulo main)@\spxentry{register()}\spxextra{en el módulo main}}

\begin{fulllineitems}
\phantomsection\label{\detokenize{modelos:main.register}}
\pysigstartsignatures
\pysiglinewithargsret
{\sphinxcode{\sphinxupquote{main.}}\sphinxbfcode{\sphinxupquote{register}}}
{\sphinxparam{\DUrole{n}{datos}\DUrole{p}{:}\DUrole{w}{ }\DUrole{n}{{\hyperref[\detokenize{modelos:main.UsuarioRegistro}]{\sphinxcrossref{UsuarioRegistro}}}}}\sphinxparamcomma \sphinxparam{\DUrole{n}{db}\DUrole{p}{:}\DUrole{w}{ }\DUrole{n}{Session}\DUrole{w}{ }\DUrole{o}{=}\DUrole{w}{ }\DUrole{default_value}{Depends(get\_db)}}}
{}
\pysigstopsignatures
\sphinxAtStartPar
Registra un nuevo usuario en la base de datos.
\begin{quote}\begin{description}
\sphinxlineitem{Parámetros}\begin{itemize}
\item {} 
\sphinxAtStartPar
\sphinxstyleliteralstrong{\sphinxupquote{datos}} ({\hyperref[\detokenize{modelos:main.UsuarioRegistro}]{\sphinxcrossref{\sphinxstyleliteralemphasis{\sphinxupquote{UsuarioRegistro}}}}}) \textendash{} Objeto que contiene el nombre de usuario y la contraseña.

\item {} 
\sphinxAtStartPar
\sphinxstyleliteralstrong{\sphinxupquote{db}} (\sphinxstyleliteralemphasis{\sphinxupquote{Session}}) \textendash{} Sesión activa de la base de datos, proporcionada por FastAPI.

\end{itemize}

\sphinxlineitem{Devuelve}
\sphinxAtStartPar
Un mensaje indicando si el usuario fue registrado exitosamente.

\sphinxlineitem{Tipo del valor devuelto}
\sphinxAtStartPar
dict

\sphinxlineitem{Muestra}
\sphinxAtStartPar
\sphinxstyleliteralstrong{\sphinxupquote{HTTPException 400}} \textendash{} Si los campos son inválidos o el nombre de usuario ya existe.

\end{description}\end{quote}

\end{fulllineitems}

\index{saludo() (en el módulo main)@\spxentry{saludo()}\spxextra{en el módulo main}}

\begin{fulllineitems}
\phantomsection\label{\detokenize{modelos:main.saludo}}
\pysigstartsignatures
\pysiglinewithargsret
{\sphinxbfcode{\sphinxupquote{\DUrole{k}{async}\DUrole{w}{ }}}\sphinxcode{\sphinxupquote{main.}}\sphinxbfcode{\sphinxupquote{saludo}}}
{}
{}
\pysigstopsignatures
\sphinxAtStartPar
Devuelve un mensaje simple para verificar que la API está activa.

\sphinxAtStartPar
Este endpoint puede utilizarse para pruebas de conectividad o para confirmar que el backend está desplegado correctamente.
\begin{quote}\begin{description}
\sphinxlineitem{Devuelve}
\sphinxAtStartPar
Mensaje de saludo indicando que la API funciona.

\sphinxlineitem{Tipo del valor devuelto}
\sphinxAtStartPar
dict

\end{description}\end{quote}

\end{fulllineitems}

\index{ver\_informe() (en el módulo main)@\spxentry{ver\_informe()}\spxextra{en el módulo main}}

\begin{fulllineitems}
\phantomsection\label{\detokenize{modelos:main.ver_informe}}
\pysigstartsignatures
\pysiglinewithargsret
{\sphinxcode{\sphinxupquote{main.}}\sphinxbfcode{\sphinxupquote{ver\_informe}}}
{\sphinxparam{\DUrole{n}{token}\DUrole{p}{:}\DUrole{w}{ }\DUrole{n}{str}}\sphinxparamcomma \sphinxparam{\DUrole{n}{db}\DUrole{p}{:}\DUrole{w}{ }\DUrole{n}{Session}\DUrole{w}{ }\DUrole{o}{=}\DUrole{w}{ }\DUrole{default_value}{Depends(get\_db)}}}
{}
\pysigstopsignatures
\sphinxAtStartPar
Devuelve los datos del informe generado a partir de un token único.

\sphinxAtStartPar
Este endpoint permite el acceso público a un informe de diagnóstico de vehículo mediante un enlace con token generado previamente. No requiere autenticación, pero valida que el token sea legítimo.
\begin{quote}\begin{description}
\sphinxlineitem{Parámetros}
\sphinxAtStartPar
\sphinxstyleliteralstrong{\sphinxupquote{token}} (\sphinxstyleliteralemphasis{\sphinxupquote{str}}) \textendash{} Token único del informe generado.

\sphinxlineitem{Devuelve}
\sphinxAtStartPar
Información del vehículo (marca, modelo, año, etc.) y lista de errores DTC.

\sphinxlineitem{Tipo del valor devuelto}
\sphinxAtStartPar
dict

\sphinxlineitem{Muestra}\begin{itemize}
\item {} 
\sphinxAtStartPar
\sphinxstyleliteralstrong{\sphinxupquote{HTTPException 400}} \textendash{} Si el token no es válido o demasiado corto.

\item {} 
\sphinxAtStartPar
\sphinxstyleliteralstrong{\sphinxupquote{HTTPException 404}} \textendash{} Si no se encuentra el informe, el vehículo o los errores asociados.

\item {} 
\sphinxAtStartPar
\sphinxstyleliteralstrong{\sphinxupquote{HTTPException 500}} \textendash{} Si ocurre un error inesperado al procesar la solicitud.

\end{itemize}

\end{description}\end{quote}

\end{fulllineitems}

\index{verificar\_password() (en el módulo main)@\spxentry{verificar\_password()}\spxextra{en el módulo main}}

\begin{fulllineitems}
\phantomsection\label{\detokenize{modelos:main.verificar_password}}
\pysigstartsignatures
\pysiglinewithargsret
{\sphinxcode{\sphinxupquote{main.}}\sphinxbfcode{\sphinxupquote{verificar\_password}}}
{\sphinxparam{\DUrole{n}{plain\_password}}\sphinxparamcomma \sphinxparam{\DUrole{n}{hashed\_password}}}
{}
\pysigstopsignatures
\sphinxAtStartPar
Verifica si una contraseña en texto plano coincide con su hash almacenado.
\begin{quote}\begin{description}
\sphinxlineitem{Parámetros}\begin{itemize}
\item {} 
\sphinxAtStartPar
\sphinxstyleliteralstrong{\sphinxupquote{password\_plano}} (\sphinxstyleliteralemphasis{\sphinxupquote{str}}) \textendash{} Contraseña proporcionada por el usuario.

\item {} 
\sphinxAtStartPar
\sphinxstyleliteralstrong{\sphinxupquote{password\_hash}} (\sphinxstyleliteralemphasis{\sphinxupquote{str}}) \textendash{} Hash almacenado en la base de datos.

\end{itemize}

\sphinxlineitem{Devuelve}
\sphinxAtStartPar
True si coinciden, False si no.

\sphinxlineitem{Tipo del valor devuelto}
\sphinxAtStartPar
bool

\end{description}\end{quote}

\end{fulllineitems}


\sphinxstepscope


\chapter{Configuración del Proyecto}
\label{\detokenize{configuracion:configuracion-del-proyecto}}\label{\detokenize{configuracion::doc}}
\sphinxAtStartPar
Este proyecto cuenta con una configuración flexible basada en variables de entorno y herramientas integradas para seguridad y mensajería.


\section{Base de Datos}
\label{\detokenize{configuracion:base-de-datos}}\begin{itemize}
\item {} 
\sphinxAtStartPar
URL: definida en \sphinxtitleref{DATABASE\_URL}.

\item {} 
\sphinxAtStartPar
Motor: MySQL con PyMySQL.

\item {} 
\sphinxAtStartPar
ORM: SQLAlchemy.

\item {} 
\sphinxAtStartPar
Sesiones: \sphinxtitleref{SessionLocal()}.

\end{itemize}


\section{Correo Electrónico}
\label{\detokenize{configuracion:correo-electronico}}\begin{itemize}
\item {} 
\sphinxAtStartPar
Envío mediante \sphinxtitleref{FastAPI\sphinxhyphen{}Mail}.

\item {} 
\sphinxAtStartPar
Configuración por variables de entorno:
\sphinxhyphen{} Usuario, contraseña, puerto, servidor SMTP.

\item {} 
\sphinxAtStartPar
Soporta TLS y SSL.

\end{itemize}


\section{Seguridad}
\label{\detokenize{configuracion:seguridad}}\begin{itemize}
\item {} 
\sphinxAtStartPar
\sphinxtitleref{JWT}: para autenticación de usuarios.

\item {} 
\sphinxAtStartPar
\sphinxtitleref{bcrypt}: para encriptar contraseñas.

\item {} 
\sphinxAtStartPar
\sphinxtitleref{OAuth2PasswordBearer}: extracción automática del token.

\end{itemize}


\section{Dependencias reutilizables}
\label{\detokenize{configuracion:dependencias-reutilizables}}\begin{itemize}
\item {} 
\sphinxAtStartPar
\sphinxtitleref{get\_db()}: genera una sesión DB por request.

\item {} 
\sphinxAtStartPar
\sphinxtitleref{obtener\_usuario\_actual()}: extrae el usuario desde el token JWT.

\end{itemize}
\index{module@\spxentry{module}!main@\spxentry{main}}\index{main@\spxentry{main}!module@\spxentry{module}}\index{get\_db() (en el módulo main)@\spxentry{get\_db()}\spxextra{en el módulo main}}\phantomsection\label{\detokenize{configuracion:module-main}}

\begin{fulllineitems}
\phantomsection\label{\detokenize{configuracion:main.get_db}}
\pysigstartsignatures
\pysiglinewithargsret
{\sphinxcode{\sphinxupquote{main.}}\sphinxbfcode{\sphinxupquote{get\_db}}}
{}
{}
\pysigstopsignatures
\sphinxAtStartPar
Dependencia de FastAPI para obtener una sesión de base de datos.

\sphinxAtStartPar
Se utiliza con \sphinxtitleref{Depends(get\_db)} para abrir una sesión, cederla al endpoint y cerrarla automáticamente.

\end{fulllineitems}



\renewcommand{\indexname}{Índice de Módulos Python}
\begin{sphinxtheindex}
\let\bigletter\sphinxstyleindexlettergroup
\bigletter{m}
\item\relax\sphinxstyleindexentry{main}\sphinxstyleindexpageref{modelos:\detokenize{module-main}}
\end{sphinxtheindex}

\renewcommand{\indexname}{Índice}
\printindex
\end{document}